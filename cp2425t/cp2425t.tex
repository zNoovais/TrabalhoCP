\documentclass[11pt, a4paper, fleqn]{article}
\usepackage{cp2425t}
\makeindex

%================= lhs2tex=====================================================%
%% ODER: format ==         = "\mathrel{==}"
%% ODER: format /=         = "\neq "
%
%
\makeatletter
\@ifundefined{lhs2tex.lhs2tex.sty.read}%
  {\@namedef{lhs2tex.lhs2tex.sty.read}{}%
   \newcommand\SkipToFmtEnd{}%
   \newcommand\EndFmtInput{}%
   \long\def\SkipToFmtEnd#1\EndFmtInput{}%
  }\SkipToFmtEnd

\newcommand\ReadOnlyOnce[1]{\@ifundefined{#1}{\@namedef{#1}{}}\SkipToFmtEnd}
\usepackage{amstext}
\usepackage{amssymb}
\usepackage{stmaryrd}
\DeclareFontFamily{OT1}{cmtex}{}
\DeclareFontShape{OT1}{cmtex}{m}{n}
  {<5><6><7><8>cmtex8
   <9>cmtex9
   <10><10.95><12><14.4><17.28><20.74><24.88>cmtex10}{}
\DeclareFontShape{OT1}{cmtex}{m}{it}
  {<-> ssub * cmtt/m/it}{}
\newcommand{\texfamily}{\fontfamily{cmtex}\selectfont}
\DeclareFontShape{OT1}{cmtt}{bx}{n}
  {<5><6><7><8>cmtt8
   <9>cmbtt9
   <10><10.95><12><14.4><17.28><20.74><24.88>cmbtt10}{}
\DeclareFontShape{OT1}{cmtex}{bx}{n}
  {<-> ssub * cmtt/bx/n}{}
\newcommand{\tex}[1]{\text{\texfamily#1}}	% NEU

\newcommand{\Sp}{\hskip.33334em\relax}


\newcommand{\Conid}[1]{\mathit{#1}}
\newcommand{\Varid}[1]{\mathit{#1}}
\newcommand{\anonymous}{\kern0.06em \vbox{\hrule\@width.5em}}
\newcommand{\plus}{\mathbin{+\!\!\!+}}
\newcommand{\bind}{\mathbin{>\!\!\!>\mkern-6.7mu=}}
\newcommand{\rbind}{\mathbin{=\mkern-6.7mu<\!\!\!<}}% suggested by Neil Mitchell
\newcommand{\sequ}{\mathbin{>\!\!\!>}}
\renewcommand{\leq}{\leqslant}
\renewcommand{\geq}{\geqslant}
\usepackage{polytable}

%mathindent has to be defined
\@ifundefined{mathindent}%
  {\newdimen\mathindent\mathindent\leftmargini}%
  {}%

\def\resethooks{%
  \global\let\SaveRestoreHook\empty
  \global\let\ColumnHook\empty}
\newcommand*{\savecolumns}[1][default]%
  {\g@addto@macro\SaveRestoreHook{\savecolumns[#1]}}
\newcommand*{\restorecolumns}[1][default]%
  {\g@addto@macro\SaveRestoreHook{\restorecolumns[#1]}}
\newcommand*{\aligncolumn}[2]%
  {\g@addto@macro\ColumnHook{\column{#1}{#2}}}

\resethooks

\newcommand{\onelinecommentchars}{\quad-{}- }
\newcommand{\commentbeginchars}{\enskip\{-}
\newcommand{\commentendchars}{-\}\enskip}

\newcommand{\visiblecomments}{%
  \let\onelinecomment=\onelinecommentchars
  \let\commentbegin=\commentbeginchars
  \let\commentend=\commentendchars}

\newcommand{\invisiblecomments}{%
  \let\onelinecomment=\empty
  \let\commentbegin=\empty
  \let\commentend=\empty}

\visiblecomments

\newlength{\blanklineskip}
\setlength{\blanklineskip}{0.66084ex}

\newcommand{\hsindent}[1]{\quad}% default is fixed indentation
\let\hspre\empty
\let\hspost\empty
\newcommand{\NB}{\textbf{NB}}
\newcommand{\Todo}[1]{$\langle$\textbf{To do:}~#1$\rangle$}

\EndFmtInput
\makeatother
%
%
%
%
%
%
% This package provides two environments suitable to take the place
% of hscode, called "plainhscode" and "arrayhscode". 
%
% The plain environment surrounds each code block by vertical space,
% and it uses \abovedisplayskip and \belowdisplayskip to get spacing
% similar to formulas. Note that if these dimensions are changed,
% the spacing around displayed math formulas changes as well.
% All code is indented using \leftskip.
%
% Changed 19.08.2004 to reflect changes in colorcode. Should work with
% CodeGroup.sty.
%
\ReadOnlyOnce{polycode.fmt}%
\makeatletter

\newcommand{\hsnewpar}[1]%
  {{\parskip=0pt\parindent=0pt\par\vskip #1\noindent}}

% can be used, for instance, to redefine the code size, by setting the
% command to \small or something alike
\newcommand{\hscodestyle}{}

% The command \sethscode can be used to switch the code formatting
% behaviour by mapping the hscode environment in the subst directive
% to a new LaTeX environment.

\newcommand{\sethscode}[1]%
  {\expandafter\let\expandafter\hscode\csname #1\endcsname
   \expandafter\let\expandafter\endhscode\csname end#1\endcsname}

% "compatibility" mode restores the non-polycode.fmt layout.

\newenvironment{compathscode}%
  {\par\noindent
   \advance\leftskip\mathindent
   \hscodestyle
   \let\\=\@normalcr
   \let\hspre\(\let\hspost\)%
   \pboxed}%
  {\endpboxed\)%
   \par\noindent
   \ignorespacesafterend}

\newcommand{\compaths}{\sethscode{compathscode}}

% "plain" mode is the proposed default.
% It should now work with \centering.
% This required some changes. The old version
% is still available for reference as oldplainhscode.

\newenvironment{plainhscode}%
  {\hsnewpar\abovedisplayskip
   \advance\leftskip\mathindent
   \hscodestyle
   \let\hspre\(\let\hspost\)%
   \pboxed}%
  {\endpboxed%
   \hsnewpar\belowdisplayskip
   \ignorespacesafterend}

\newenvironment{oldplainhscode}%
  {\hsnewpar\abovedisplayskip
   \advance\leftskip\mathindent
   \hscodestyle
   \let\\=\@normalcr
   \(\pboxed}%
  {\endpboxed\)%
   \hsnewpar\belowdisplayskip
   \ignorespacesafterend}

% Here, we make plainhscode the default environment.

\newcommand{\plainhs}{\sethscode{plainhscode}}
\newcommand{\oldplainhs}{\sethscode{oldplainhscode}}
\plainhs

% The arrayhscode is like plain, but makes use of polytable's
% parray environment which disallows page breaks in code blocks.

\newenvironment{arrayhscode}%
  {\hsnewpar\abovedisplayskip
   \advance\leftskip\mathindent
   \hscodestyle
   \let\\=\@normalcr
   \(\parray}%
  {\endparray\)%
   \hsnewpar\belowdisplayskip
   \ignorespacesafterend}

\newcommand{\arrayhs}{\sethscode{arrayhscode}}

% The mathhscode environment also makes use of polytable's parray 
% environment. It is supposed to be used only inside math mode 
% (I used it to typeset the type rules in my thesis).

\newenvironment{mathhscode}%
  {\parray}{\endparray}

\newcommand{\mathhs}{\sethscode{mathhscode}}

% texths is similar to mathhs, but works in text mode.

\newenvironment{texthscode}%
  {\(\parray}{\endparray\)}

\newcommand{\texths}{\sethscode{texthscode}}

% The framed environment places code in a framed box.

\def\codeframewidth{\arrayrulewidth}
\RequirePackage{calc}

\newenvironment{framedhscode}%
  {\parskip=\abovedisplayskip\par\noindent
   \hscodestyle
   \arrayrulewidth=\codeframewidth
   \tabular{@{}|p{\linewidth-2\arraycolsep-2\arrayrulewidth-2pt}|@{}}%
   \hline\framedhslinecorrect\\{-1.5ex}%
   \let\endoflinesave=\\
   \let\\=\@normalcr
   \(\pboxed}%
  {\endpboxed\)%
   \framedhslinecorrect\endoflinesave{.5ex}\hline
   \endtabular
   \parskip=\belowdisplayskip\par\noindent
   \ignorespacesafterend}

\newcommand{\framedhslinecorrect}[2]%
  {#1[#2]}

\newcommand{\framedhs}{\sethscode{framedhscode}}

% The inlinehscode environment is an experimental environment
% that can be used to typeset displayed code inline.

\newenvironment{inlinehscode}%
  {\(\def\column##1##2{}%
   \let\>\undefined\let\<\undefined\let\\\undefined
   \newcommand\>[1][]{}\newcommand\<[1][]{}\newcommand\\[1][]{}%
   \def\fromto##1##2##3{##3}%
   \def\nextline{}}{\) }%

\newcommand{\inlinehs}{\sethscode{inlinehscode}}

% The joincode environment is a separate environment that
% can be used to surround and thereby connect multiple code
% blocks.

\newenvironment{joincode}%
  {\let\orighscode=\hscode
   \let\origendhscode=\endhscode
   \def\endhscode{\def\hscode{\endgroup\def\@currenvir{hscode}\\}\begingroup}
   %\let\SaveRestoreHook=\empty
   %\let\ColumnHook=\empty
   %\let\resethooks=\empty
   \orighscode\def\hscode{\endgroup\def\@currenvir{hscode}}}%
  {\origendhscode
   \global\let\hscode=\orighscode
   \global\let\endhscode=\origendhscode}%

\makeatother
\EndFmtInput
%




%------------------------------------------------------------------------------%

%====== DEFINIR GRUPO E ELEMENTOS =============================================%

\group{G02}
\studentA{108395}{José Pedro Flores Novais }
\studentB{108840}{Guilherme Dall'Agnol }
\studentC{108653}{João Manuel Pinto Alves }

%==============================================================================%

\begin{document}

\sffamily
\setlength{\parindent}{0em}
\emergencystretch 3em
\renewcommand{\baselinestretch}{1.25} 
\begin{titlepage}
\includegraphics[width=26mm]{cp2425t_media/UM.jpg}\includegraphics[width=26mm]{cp2425t_media/EE.jpg}

\vspace{7mm}
\leading{17pt}
{\Large
\textbf{Universidade do Minho}
\\
{\fontseries{l}\selectfont
Escola de Engenharia
}}

\vspace{50mm}
\leading{27pt}
{\huge
\textbf{Cálculo de Programas}
\\
Trabalho Prático (2024/25)}
\\
%
\\
{\LARGE
Lic.\ em Ciências da Computação}

\vspace{50mm}
\leading{15pt}
{\Large
\begin{tabularx}{\textwidth}{@{}p{20mm} X}
    \multicolumn{2}{@{}l}{\textbf{Grupo \thegroup}} \\
    \multicolumn{2}{@{}l}{} \\
    \thestudentA \\
    \thestudentB \\
    \thestudentC \\
    %\thestudentD \\
\end{tabularx}
}

\vspace*{\fill}
{\large \myear}
\end{titlepage}

\pagestyle{pagestyle}
\setlength{\parindent}{1em}
\newgeometry{left=25mm,right=20mm,top=25mm,bottom=25mm}

\section*{Preâmbulo}

Na UC de \CP\ pretende-se ensinar a progra\-mação de computadores como uma disciplina
científica. Para isso parte-se de um repertório de \emph{combinadores} que
formam uma álgebra da programação % (conjunto de leis universais e seus corolários)
e usam-se esses combinadores para construir programas \emph{composicionalmente},
isto é, agregando programas já existentes.

Na sequência pedagógica dos planos de estudo dos cursos que têm esta disciplina,
opta-se pela aplicação deste método à programação em \Haskell\ (sem prejuízo
da sua aplicação a outras linguagens funcionais). Assim, o presente trabalho
prático coloca os alunos perante problemas concretos que deverão ser implementados
em \Haskell. Há ainda um outro objectivo: o de ensinar a documentar programas,
a validá-los e a produzir textos técnico-científicos de qualidade.

Antes de abordarem os problemas propostos no trabalho, os grupos devem ler
com atenção o anexo \ref{sec:documentacao} onde encontrarão as instruções
relativas ao \emph{software} a instalar, etc.

Valoriza-se a escrita de \emph{pouco} código que corresponda a soluções simples
e elegantes que utilizem os combinadores de ordem superior estudados na disciplina.


\Problema
Esta questão aborda um problema que é conhecido pela designação '\emph{Container
With Most Water}' e que se formula facilmente através do exemplo da figura
seguinte:

	\histogramaA \label{fig:histogramaA}

\noindent
A figura mostra a sequência de números
\begin{hscode}\SaveRestoreHook
\column{B}{@{}>{\hspre}l<{\hspost}@{}}%
\column{E}{@{}>{\hspre}l<{\hspost}@{}}%
\>[B]{}\Varid{hghts}\mathrel{=}[\mskip1.5mu \mathrm{1},\mathrm{8},\mathrm{6},\mathrm{2},\mathrm{5},\mathrm{4},\mathrm{8},\mathrm{3},\mathrm{7}\mskip1.5mu]{}\<[E]%
\ColumnHook
\end{hscode}\resethooks
representada sob a forma de um histograma. O que se pretende é obter a maior
área rectangular delimitada por duas barras do histograma, área essa marcada
a azul na figura. (A ``metáfora'' \emph{container with most water} sugere que
as barras selecionadas delimitam um \emph{container} com água.)

Pretende-se definida como um catamorfismo, anamorfismo ou hilomorfismo uma
função em \Haskell
\begin{hscode}\SaveRestoreHook
\column{B}{@{}>{\hspre}l<{\hspost}@{}}%
\column{E}{@{}>{\hspre}l<{\hspost}@{}}%
\>[B]{}\Varid{mostwater}\mathbin{::}[\mskip1.5mu \Conid{Int}\mskip1.5mu]\to \Conid{Int}{}\<[E]%
\ColumnHook
\end{hscode}\resethooks
que deverá dar essa área. (No exemplo acima tem-se \ensuremath{\Varid{mostwater}\;[\mskip1.5mu \mathrm{1},\mathrm{8},\mathrm{6},\mathrm{2},\mathrm{5},\mathrm{4},\mathrm{8},\mathrm{3},\mathrm{7}\mskip1.5mu]\mathrel{=}\mathrm{49}}.)
A resolução desta questão deverá ser acompanhada de diagramas elucidativos.

\Problema

Um dos problemas prementes da Computação na actualidade é conseguir, por
engenharia reversa, interpretar as redes neuronais (\NN{RN}) geradas artificialmente
sob a forma de algoritmos compreensíveis por humanos.

Já foram dados passos que, nesse sentido, explicam vários padrões de \NN{RN}s
em termos de combinadores funcionais \cite{Co15}. Em particular, já se mostrou
como as {\RNN}s (\emph{Recurrent Neural Networks}) podem ser vistas como
instâncias de \emph{accumulating maps}, que em \Haskell\ correspondem às
funções \ensuremath{\mapAccumR } e \ensuremath{\mapAccumL }, conforme o sentido em que a acumulação
se verifica (cf.\ figura \ref{fig:RNNAsMapAccumR}).

\RNNAsMapAccumR

A \RNN\ que a figura \ref{fig:RNNAsMapAccumR} mostra diz-se \emph{'one-to-one'}
\cite{Ka15}. Há contudo padrões de {\RNN}s mais gerais: por exemplo, o padrão
\emph{'many-to-one'} que se mostra na figura \ref{fig:RNNs} extraída
de  \cite{Ka15}.

Se \ensuremath{\mapAccumR } e \ensuremath{\mapAccumL } juntam \ensuremath{\Varid{maps}} com \ensuremath{\Varid{folds}}, pretendemos agora
combinadores que a isso acrescentem \ensuremath{\Varid{filter}}, por forma a selecionar que
etapas da computação geram ou não \emph{outputs} --- obtendo-se assim o efeito
\emph{'many-to-one'}. Ter-se-á, para esse efeito:

\begin{hscode}\SaveRestoreHook
\column{B}{@{}>{\hspre}l<{\hspost}@{}}%
\column{E}{@{}>{\hspre}l<{\hspost}@{}}%
\>[B]{}\Varid{mapAccumRfilter}\mathbin{::}((\Varid{a},\Varid{s})\to \Conid{Bool})\to ((\Varid{a},\Varid{s})\to (\Varid{c},\Varid{s}))\to ([\mskip1.5mu \Varid{a}\mskip1.5mu],\Varid{s})\to ([\mskip1.5mu \Varid{c}\mskip1.5mu],\Varid{s}){}\<[E]%
\\
\>[B]{}\Varid{mapAccumLfilter}\mathbin{::}((\Varid{a},\Varid{s})\to \Conid{Bool})\to ((\Varid{a},\Varid{s})\to (\Varid{c},\Varid{s}))\to ([\mskip1.5mu \Varid{a}\mskip1.5mu],\Varid{s})\to ([\mskip1.5mu \Varid{c}\mskip1.5mu],\Varid{s}){}\<[E]%
\ColumnHook
\end{hscode}\resethooks

Pretende-se a implementação de \ensuremath{\Varid{mapAccumRfilter}} e \ensuremath{\Varid{mapAccumLfilter}} sob a forma de ana / cata ou hilomorfismos em \Haskell, acompanhadas por diagramas.

Como caso de uso, sugere-se o que se dá no anexo \ref{sec:karpathy} que, inspirado em \cite{Ka15}, recorre à biblioteca \DataMatrix.

\Problema

Umas das fórmulas conhecidas para calcular o número \ensuremath{\pi } é a que se segue,
\begin{eqnarray}
	\ensuremath{\pi } = \sum_{n=0}^\infty \frac{(n!)^2 {2^{n+1}}}{(2n+1)!}
	\label{eq:pi}
\end{eqnarray}
correspondente à função \ensuremath{\pi_{\mathit{calc}}} cuja implementação em Haskell, paramétrica em \ensuremath{\Varid{n}}, é dada no anexo \ref{sec:codigo}.

Pretende-se uma implementação eficiente de (\ref{eq:pi}) que, derivada por recursividade mútua,
não calcule factoriais nenhuns:
\begin{hscode}\SaveRestoreHook
\column{B}{@{}>{\hspre}l<{\hspost}@{}}%
\column{E}{@{}>{\hspre}l<{\hspost}@{}}%
\>[B]{}\pi_{\mathit{loop}}\mathrel{=}\cdots \comp \for{\Varid{loop}}\ {\Varid{inic}}\;\mathbf{where}\;\cdots {}\<[E]%
\ColumnHook
\end{hscode}\resethooks
\textbf{Sugestão}: recomenda-se a \textbf{regra prática} que se dá no anexo \ref{sec:mr}
para problemas deste género, que podem envolver várias decomposições por recursividade
mútua em \ensuremath{\N_0}.

\RNNs

\Problema
Considere-se a matriz e o vector que se seguem:
\begin{eqnarray}
mat&=&\begin{bmatrix}
      1 & -1 & 2 \\
      0 & -3 & 1
      \end{bmatrix}
      \label{eq:mat}
\\
vec&=&\begin{bmatrix}
      2  \\
      1 \\
      0
      \end{bmatrix}
      \label{eq:vec}
\end{eqnarray}
Em \Haskell, podemos tornar explícito o espaço vectorial a que (\ref{eq:vec}) pertence definindo-o da forma seguinte,
\begin{hscode}\SaveRestoreHook
\column{B}{@{}>{\hspre}l<{\hspost}@{}}%
\column{E}{@{}>{\hspre}l<{\hspost}@{}}%
\>[B]{}\Varid{vec}\mathbin{::}\Conid{Vec}\;\Conid{X}{}\<[E]%
\\
\>[B]{}\Varid{vec}\mathrel{=}\Conid{V}\;[\mskip1.5mu (X_1 ,\mathrm{2}),(X_2 ,\mathrm{1}),(X_3 ,\mathrm{0})\mskip1.5mu]{}\<[E]%
\ColumnHook
\end{hscode}\resethooks
assumindo definido o tipo
\begin{hscode}\SaveRestoreHook
\column{B}{@{}>{\hspre}l<{\hspost}@{}}%
\column{E}{@{}>{\hspre}l<{\hspost}@{}}%
\>[B]{}\mathbf{data}\;\Conid{Vec}\;\Varid{a}\mathrel{=}\Conid{V}\;\{\mskip1.5mu \Varid{outV}\mathbin{::}[\mskip1.5mu (\Varid{a},\Conid{Int})\mskip1.5mu]\mskip1.5mu\}\;\mathbf{deriving}\;(\Conid{Ord}){}\<[E]%
\ColumnHook
\end{hscode}\resethooks
e o ``tipo-dimensão'':
\begin{hscode}\SaveRestoreHook
\column{B}{@{}>{\hspre}l<{\hspost}@{}}%
\column{E}{@{}>{\hspre}l<{\hspost}@{}}%
\>[B]{}\mathbf{data}\;\Conid{X}\mathrel{=}X_1 \mid X_2 \mid X_3 \;\mathbf{deriving}\;(\Conid{Eq},\Conid{Show},\Conid{Ord}){}\<[E]%
\ColumnHook
\end{hscode}\resethooks

Da mesma forma que \emph{tipamos} \ensuremath{\Varid{vec}}, também o podemos fazer para a matrix \ensuremath{\Varid{mat}} (\ref{eq:mat}), cujas colunas podem ser indexadas por \ensuremath{\Conid{X}} também e as linhas por \ensuremath{\Conid{Bool}}, por exemplo:
\begin{hscode}\SaveRestoreHook
\column{B}{@{}>{\hspre}l<{\hspost}@{}}%
\column{E}{@{}>{\hspre}l<{\hspost}@{}}%
\>[B]{}\Varid{mat}\mathbin{::}\Conid{X}\to \Conid{Vec}\;\Conid{Bool}{}\<[E]%
\\
\>[B]{}\Varid{mat}\;X_1 \mathrel{=}\Conid{V}\;[\mskip1.5mu (\Conid{False},\mathrm{1}),(\Conid{True},\mathrm{0})\mskip1.5mu]{}\<[E]%
\\
\>[B]{}\Varid{mat}\;X_2 \mathrel{=}\Conid{V}\;[\mskip1.5mu (\Conid{False},\mathbin{-}\mathrm{1}),(\Conid{True},\mathbin{-}\mathrm{3})\mskip1.5mu]{}\<[E]%
\\
\>[B]{}\Varid{mat}\;X_3 \mathrel{=}\Conid{V}\;[\mskip1.5mu (\Conid{False},\mathrm{2}),(\Conid{True},\mathrm{1})\mskip1.5mu]{}\<[E]%
\ColumnHook
\end{hscode}\resethooks
Quer dizer, matrizes podem ser encaradas como funções que dão vectores como
resultado. Mais ainda, a multiplicação de \ensuremath{\Varid{mat}} por \ensuremath{\Varid{vec}} pode ser obtida
correndo, simplesmente
\begin{hscode}\SaveRestoreHook
\column{B}{@{}>{\hspre}l<{\hspost}@{}}%
\column{E}{@{}>{\hspre}l<{\hspost}@{}}%
\>[B]{}\Varid{vec'}\mathrel{=}\Varid{vec}\bind \Varid{mat}{}\<[E]%
\ColumnHook
\end{hscode}\resethooks
obtendo-se \ensuremath{\Varid{vec'}\mathrel{=}\Conid{V}\;[\mskip1.5mu (\Conid{False},\mathrm{1}),(\Conid{True},\mathbin{-}\mathrm{3})\mskip1.5mu]} do tipo \ensuremath{\Conid{Vec}\;\Conid{Bool}}.
Finalmente, se for dada a matrix
\begin{hscode}\SaveRestoreHook
\column{B}{@{}>{\hspre}l<{\hspost}@{}}%
\column{11}{@{}>{\hspre}l<{\hspost}@{}}%
\column{E}{@{}>{\hspre}l<{\hspost}@{}}%
\>[B]{}\Varid{neg}\mathbin{::}\Conid{Bool}\to \Conid{Vec}\;\Conid{Bool}{}\<[E]%
\\
\>[B]{}\Varid{neg}\;\Conid{False}\mathrel{=}\Conid{V}\;[\mskip1.5mu (\Conid{False},\mathrm{0}),(\Conid{True},\mathrm{1})\mskip1.5mu]{}\<[E]%
\\
\>[B]{}\Varid{neg}\;\Conid{True}{}\<[11]%
\>[11]{}\mathrel{=}\Conid{V}\;[\mskip1.5mu (\Conid{False},\mathrm{1}),(\Conid{True},\mathrm{0})\mskip1.5mu]{}\<[E]%
\ColumnHook
\end{hscode}\resethooks
então a multiplicação de \ensuremath{\Varid{neg}} por \ensuremath{\Varid{mat}} mais não será que a matriz
\begin{hscode}\SaveRestoreHook
\column{B}{@{}>{\hspre}l<{\hspost}@{}}%
\column{E}{@{}>{\hspre}l<{\hspost}@{}}%
\>[B]{}\Varid{neg}\mathbin{\bullet}\Varid{mat}{}\<[E]%
\ColumnHook
\end{hscode}\resethooks
também do tipo \ensuremath{\Conid{X}\to \Conid{Vec}\;\Conid{Bool}}.

Obtém-se assim uma \emph{álgebra linear tipada}. Contudo, para isso é preciso
mostrar que \ensuremath{\Conid{Vec}} é um \textbf{mónade}, e é esse o tema desta questão, em duas partes:
\begin{itemize}
\item	
Instanciar \ensuremath{\Conid{Vec}} na class \ensuremath{\Conid{Functor}} em \Haskell:
\begin{hscode}\SaveRestoreHook
\column{B}{@{}>{\hspre}l<{\hspost}@{}}%
\column{5}{@{}>{\hspre}l<{\hspost}@{}}%
\column{E}{@{}>{\hspre}l<{\hspost}@{}}%
\>[B]{}\mathbf{instance}\;\Conid{Functor}\;\Conid{Vec}\;\mathbf{where}{}\<[E]%
\\
\>[B]{}\hsindent{5}{}\<[5]%
\>[5]{}\mathsf{fmap}\;\Varid{f}\mathrel{=}\mathbin{....}{}\<[E]%
\ColumnHook
\end{hscode}\resethooks
\item	
Instanciar \ensuremath{\Conid{Vec}} na class \ensuremath{\Conid{Monad}} em \Haskell:
\begin{hscode}\SaveRestoreHook
\column{B}{@{}>{\hspre}l<{\hspost}@{}}%
\column{4}{@{}>{\hspre}l<{\hspost}@{}}%
\column{E}{@{}>{\hspre}l<{\hspost}@{}}%
\>[B]{}\mathbf{instance}\;\Conid{Monad}\;\Conid{Vec}\;\mathbf{where}{}\<[E]%
\\
\>[B]{}\hsindent{4}{}\<[4]%
\>[4]{}\Varid{x}\bind \Varid{f}\mathrel{=}\mathbin{....}{}\<[E]%
\\
\>[B]{}\hsindent{4}{}\<[4]%
\>[4]{}\Varid{return}\;\Varid{a}\mathrel{=}\mathbin{...}{}\<[E]%
\ColumnHook
\end{hscode}\resethooks
\end{itemize} 

\part*{Anexos}

\appendix

\section{Natureza do trabalho a realizar}
\label{sec:documentacao}
Este trabalho teórico-prático deve ser realizado por grupos de 3 alunos.
Os detalhes da avaliação (datas para submissão do relatório e sua defesa
oral) são os que forem publicados na \cp{página da disciplina} na \emph{internet}.

Recomenda-se uma abordagem participativa dos membros do grupo em \textbf{todos}
os exercícios do trabalho, para assim poderem responder a qualquer questão
colocada na \emph{defesa oral} do relatório.

Para cumprir de forma integrada os objectivos do trabalho vamos recorrer
a uma técnica de programa\-ção dita ``\litp{literária}'' \cite{Kn92}, cujo
princípio base é o seguinte:
%
\begin{quote}\em
	Um programa e a sua documentação devem coincidir.
\end{quote}
%
Por outras palavras, o \textbf{código fonte} e a \textbf{documentação} de um
programa deverão estar no mesmo ficheiro.

O ficheiro \texttt{cp2425t.pdf} que está a ler é já um exemplo de
\litp{programação literária}: foi gerado a partir do texto fonte
\texttt{cp2425t.lhs}\footnote{O sufixo `lhs' quer dizer
\emph{\lhaskell{literate Haskell}}.} que encontrará no \MaterialPedagogico\
desta disciplina des\-com\-pactando o ficheiro \texttt{cp2425t.zip}.

Como se mostra no esquema abaixo, de um único ficheiro (\ensuremath{\Varid{lhs}})
gera-se um PDF ou faz-se a interpretação do código \Haskell\ que ele inclui:

	\esquema

Vê-se assim que, para além do \GHCi, serão necessários os executáveis \PdfLatex\ e
\LhsToTeX. Para facilitar a instalação e evitar problemas de versões e
conflitos com sistemas operativos, é recomendado o uso do \Docker\ tal como
a seguir se descreve.

\section{Docker} \label{sec:docker}

Recomenda-se o uso do \container\ cuja imagem é gerada pelo \Docker\ a partir do ficheiro
\texttt{Dockerfile} que se encontra na diretoria que resulta de descompactar
\texttt{cp2425t.zip}. Este \container\ deverá ser usado na execução
do \GHCi\ e dos comandos relativos ao \Latex. (Ver também a \texttt{Makefile}
que é disponibilizada.)

Após \href{https://docs.docker.com/engine/install/}{instalar o Docker} e
descarregar o referido zip com o código fonte do trabalho,
basta executar os seguintes comandos:
\begin{Verbatim}[fontsize=\small]
    $ docker build -t cp2425t .
    $ docker run -v ${PWD}:/cp2425t -it cp2425t
\end{Verbatim}
\textbf{NB}: O objetivo é que o container\ seja usado \emph{apenas} 
para executar o \GHCi\ e os comandos relativos ao \Latex.
Deste modo, é criado um \textit{volume} (cf.\ a opção \texttt{-v \$\{PWD\}:/cp2425t}) 
que permite que a diretoria em que se encontra na sua máquina local 
e a diretoria \texttt{/cp2425t} no \container\ sejam partilhadas.

Pretende-se então que visualize/edite os ficheiros na sua máquina local e que
os compile no \container, executando:
\begin{Verbatim}[fontsize=\small]
    $ lhs2TeX cp2425t.lhs > cp2425t.tex
    $ pdflatex cp2425t
\end{Verbatim}
\LhsToTeX\ é o pre-processador que faz ``pretty printing'' de código \Haskell\
em \Latex\ e que faz parte já do \container. Alternativamente, basta executar
\begin{Verbatim}[fontsize=\small]
    $ make
\end{Verbatim}
para obter o mesmo efeito que acima.

Por outro lado, o mesmo ficheiro \texttt{cp2425t.lhs} é executável e contém
o ``kit'' básico, escrito em \Haskell, para realizar o trabalho. Basta executar
\begin{Verbatim}[fontsize=\small]
    $ ghci cp2425t.lhs
\end{Verbatim}

\noindent Abra o ficheiro \texttt{cp2425t.lhs} no seu editor de texto preferido
e verifique que assim é: todo o texto que se encontra dentro do ambiente
\begin{quote}\small\tt
\text{\ttfamily \char92{}begin\char123{}code\char125{}}
\\ ... \\
\text{\ttfamily \char92{}end\char123{}code\char125{}}
\end{quote}
é seleccionado pelo \GHCi\ para ser executado.

\section{Em que consiste o TP}

Em que consiste, então, o \emph{relatório} a que se referiu acima?
É a edição do texto que está a ser lido, preenchendo o anexo \ref{sec:resolucao}
com as respostas. O relatório deverá conter ainda a identificação dos membros
do grupo de trabalho, no local respectivo da folha de rosto.

Para gerar o PDF integral do relatório deve-se ainda correr os comando seguintes,
que actualizam a bibliografia (com \Bibtex) e o índice remissivo (com \Makeindex),
\begin{Verbatim}[fontsize=\small]
    $ bibtex cp2425t.aux
    $ makeindex cp2425t.idx
\end{Verbatim}
e recompilar o texto como acima se indicou. (Como já se disse, pode fazê-lo
correndo simplesmente \texttt{make} no \container.)

No anexo \ref{sec:codigo} disponibiliza-se algum código \Haskell\ relativo
aos problemas que são colocados. Esse anexo deverá ser consultado e analisado
à medida que isso for necessário.

Deve ser feito uso da \litp{programação literária} para documentar bem o código que se
desenvolver, em particular fazendo diagramas explicativos do que foi feito e
tal como se explica no anexo \ref{sec:diagramas} que se segue.

\section{Como exprimir cálculos e diagramas em LaTeX/lhs2TeX} \label{sec:diagramas}

Como primeiro exemplo, estudar o texto fonte (\lhstotex{lhs}) do que está a ler\footnote{
Procure e.g.\ por \texttt{"sec:diagramas"}.} onde se obtém o efeito seguinte:\footnote{Exemplos
tirados de \cite{Ol05}.}
\begin{eqnarray*}
\start
\ensuremath{\Varid{id}\mathrel{=}\conj{\Varid{f}}{\Varid{g}}}
\just\equiv{ universal property }
\ensuremath{\begin{lcbr}\p1\comp \Varid{id}\mathrel{=}\Varid{f}\\\p2\comp \Varid{id}\mathrel{=}\Varid{g}\end{lcbr}}
\just\equiv{ identity }
\ensuremath{\begin{lcbr}\p1\mathrel{=}\Varid{f}\\\p2\mathrel{=}\Varid{g}\end{lcbr}}
\qed
\end{eqnarray*}

Os diagramas podem ser produzidos recorrendo à \emph{package} \Xymatrix, por exemplo:
\begin{eqnarray*}
\xymatrix@C=2cm{
    \ensuremath{\N_0}
           \ar[d]_-{\ensuremath{\cataNat{\Varid{g}}}}
&
    \ensuremath{\mathrm{1}\mathbin{+}\N_0}
           \ar[d]^{\ensuremath{\Varid{id}\mathbin{+}\cataNat{\Varid{g}}}}
           \ar[l]_-{\ensuremath{\mathsf{in}}}
\\
     \ensuremath{\Conid{B}}
&
     \ensuremath{\mathrm{1}\mathbin{+}\Conid{B}}
           \ar[l]^-{\ensuremath{\Varid{g}}}
}
\end{eqnarray*}

\section{Regra prática para a recursividade mútua em \ensuremath{\N_0}}\label{sec:mr}

Nesta disciplina estudou-se como fazer \pd{programação dinâmica} por cálculo,
recorrendo à lei de recursividade mútua.\footnote{Lei (\ref{eq:fokkinga})
em \cite{Ol05}, página \pageref{eq:fokkinga}.}

Para o caso de funções sobre os números naturais (\ensuremath{\N_0}, com functor \ensuremath{\fun F \;\Conid{X}\mathrel{=}\mathrm{1}\mathbin{+}\Conid{X}}) pode derivar-se da lei que foi estudada uma
	\emph{regra de algibeira}
	\label{pg:regra}
que se pode ensinar a programadores que não tenham estudado
\cp{Cálculo de Programas}. Apresenta-se de seguida essa regra, tomando como
exemplo o cálculo do ciclo-\textsf{for} que implementa a função de Fibonacci,
recordar o sistema:
\begin{hscode}\SaveRestoreHook
\column{B}{@{}>{\hspre}l<{\hspost}@{}}%
\column{E}{@{}>{\hspre}l<{\hspost}@{}}%
\>[B]{}\Varid{fib}\;\mathrm{0}\mathrel{=}\mathrm{1}{}\<[E]%
\\
\>[B]{}\Varid{fib}\;(\Varid{n}\mathbin{+}\mathrm{1})\mathrel{=}\Varid{f}\;\Varid{n}{}\<[E]%
\\[\blanklineskip]%
\>[B]{}\Varid{f}\;\mathrm{0}\mathrel{=}\mathrm{1}{}\<[E]%
\\
\>[B]{}\Varid{f}\;(\Varid{n}\mathbin{+}\mathrm{1})\mathrel{=}\Varid{fib}\;\Varid{n}\mathbin{+}\Varid{f}\;\Varid{n}{}\<[E]%
\ColumnHook
\end{hscode}\resethooks
Obter-se-á de imediato
\begin{hscode}\SaveRestoreHook
\column{B}{@{}>{\hspre}l<{\hspost}@{}}%
\column{4}{@{}>{\hspre}l<{\hspost}@{}}%
\column{E}{@{}>{\hspre}l<{\hspost}@{}}%
\>[B]{}\Varid{fib'}\mathrel{=}\p1\comp \for{\Varid{loop}}\ {\Varid{init}}\;\mathbf{where}{}\<[E]%
\\
\>[B]{}\hsindent{4}{}\<[4]%
\>[4]{}\Varid{loop}\;(\Varid{fib},\Varid{f})\mathrel{=}(\Varid{f},\Varid{fib}\mathbin{+}\Varid{f}){}\<[E]%
\\
\>[B]{}\hsindent{4}{}\<[4]%
\>[4]{}\Varid{init}\mathrel{=}(\mathrm{1},\mathrm{1}){}\<[E]%
\ColumnHook
\end{hscode}\resethooks
usando as regras seguintes:
\begin{itemize}
\item	O corpo do ciclo \ensuremath{\Varid{loop}} terá tantos argumentos quanto o número de funções mutuamente recursivas.
\item	Para as variáveis escolhem-se os próprios nomes das funções, pela ordem
que se achar conveniente.\footnote{Podem obviamente usar-se outros símbolos, mas numa primeira leitura
dá jeito usarem-se tais nomes.}
\item	Para os resultados vão-se buscar as expressões respectivas, retirando a variável \ensuremath{\Varid{n}}.
\item	Em \ensuremath{\Varid{init}} coleccionam-se os resultados dos casos de base das funções, pela mesma ordem.
\end{itemize}
Mais um exemplo, envolvendo polinómios do segundo grau $ax^2 + b x + c$ em \ensuremath{\N_0}.
Seguindo o método estudado nas aulas\footnote{Secção 3.17 de \cite{Ol05} e tópico
\href{https://www4.di.uminho.pt/~jno/media/cp/}{Recursividade mútua} nos vídeos de apoio às aulas teóricas.},
de $f\ x = a x^2 + b x + c$ derivam-se duas funções mutuamente recursivas:
\begin{hscode}\SaveRestoreHook
\column{B}{@{}>{\hspre}l<{\hspost}@{}}%
\column{E}{@{}>{\hspre}l<{\hspost}@{}}%
\>[B]{}\Varid{f}\;\mathrm{0}\mathrel{=}\Varid{c}{}\<[E]%
\\
\>[B]{}\Varid{f}\;(\Varid{n}\mathbin{+}\mathrm{1})\mathrel{=}\Varid{f}\;\Varid{n}\mathbin{+}\Varid{k}\;\Varid{n}{}\<[E]%
\\[\blanklineskip]%
\>[B]{}\Varid{k}\;\mathrm{0}\mathrel{=}\Varid{a}\mathbin{+}\Varid{b}{}\<[E]%
\\
\>[B]{}\Varid{k}\;(\Varid{n}\mathbin{+}\mathrm{1})\mathrel{=}\Varid{k}\;\Varid{n}\mathbin{+}\mathrm{2}\;\Varid{a}{}\<[E]%
\ColumnHook
\end{hscode}\resethooks
Seguindo a regra acima, calcula-se de imediato a seguinte implementação, em Haskell:
\begin{hscode}\SaveRestoreHook
\column{B}{@{}>{\hspre}l<{\hspost}@{}}%
\column{3}{@{}>{\hspre}l<{\hspost}@{}}%
\column{E}{@{}>{\hspre}l<{\hspost}@{}}%
\>[B]{}\Varid{f'}\;\Varid{a}\;\Varid{b}\;\Varid{c}\mathrel{=}\p1\comp \for{\Varid{loop}}\ {\Varid{init}}\;\mathbf{where}{}\<[E]%
\\
\>[B]{}\hsindent{3}{}\<[3]%
\>[3]{}\Varid{loop}\;(\Varid{f},\Varid{k})\mathrel{=}(\Varid{f}\mathbin{+}\Varid{k},\Varid{k}\mathbin{+}\mathrm{2}\mathbin{*}\Varid{a}){}\<[E]%
\\
\>[B]{}\hsindent{3}{}\<[3]%
\>[3]{}\Varid{init}\mathrel{=}(\Varid{c},\Varid{a}\mathbin{+}\Varid{b}){}\<[E]%
\ColumnHook
\end{hscode}\resethooks

\section{Código fornecido}\label{sec:codigo}

\subsection*{Problema 1}
Teste relativo à figura da página \pageref{fig:histogramaA}:
\begin{hscode}\SaveRestoreHook
\column{B}{@{}>{\hspre}l<{\hspost}@{}}%
\column{E}{@{}>{\hspre}l<{\hspost}@{}}%
\>[B]{}test_{1} \mathrel{=}\Varid{mostwater}\;\Varid{hghts}{}\<[E]%
\ColumnHook
\end{hscode}\resethooks

\subsection*{Problema 2}\label{sec:karpathy}

\RNNcharseq

Testes relativos a \ensuremath{\Varid{mapAccumLfilter}} e \ensuremath{\Varid{mapAccumRfilter}} em geral (comparar os \emph{outputs})

\begin{hscode}\SaveRestoreHook
\column{B}{@{}>{\hspre}l<{\hspost}@{}}%
\column{E}{@{}>{\hspre}l<{\hspost}@{}}%
\>[B]{}test_{2a} \mathrel{=}\Varid{mapAccumLfilter}\;((\mathbin{>}\mathrm{10})\comp \p1)\;\Varid{f}\;(\Varid{odds}\;\mathrm{12},\mathrm{0}){}\<[E]%
\\
\>[B]{}test_{2b} \mathrel{=}\Varid{mapAccumRfilter}\;((\mathbin{>}\mathrm{10})\comp \p1)\;\Varid{f}\;(\Varid{odds}\;\mathrm{12},\mathrm{0}){}\<[E]%
\ColumnHook
\end{hscode}\resethooks
onde:
\begin{hscode}\SaveRestoreHook
\column{B}{@{}>{\hspre}l<{\hspost}@{}}%
\column{E}{@{}>{\hspre}l<{\hspost}@{}}%
\>[B]{}\Varid{odds}\;\Varid{n}\mathrel{=}\map \;((\mathrm{1}\mathbin{+})\comp (\mathrm{2}\mathbin{*}))\;[\mskip1.5mu \mathrm{0}\mathinner{\ldotp\ldotp}\Varid{n}\mathbin{-}\mathrm{1}\mskip1.5mu]{}\<[E]%
\\
\>[B]{}\Varid{f}\;(\Varid{a},\Varid{s})\mathrel{=}(\Varid{s},\Varid{a}\mathbin{+}\Varid{s}){}\<[E]%
\ColumnHook
\end{hscode}\resethooks
Teste 
\begin{hscode}\SaveRestoreHook
\column{B}{@{}>{\hspre}l<{\hspost}@{}}%
\column{E}{@{}>{\hspre}l<{\hspost}@{}}%
\>[B]{}test_{2c} \mathrel{=}\Varid{mapAccumLfilter}\;\Varid{true}\;\Varid{step}\;([\mskip1.5mu x_1 ,x_2 ,x_3 ,x_4 \mskip1.5mu],h_0 ){}\<[E]%
\ColumnHook
\end{hscode}\resethooks
baseado no exemplo de Karpathy \cite{Ka15} que a figura \ref{fig:RNNcharseq} mostra, usando os dados seguintes:
\begin{itemize}
\item	Estado inicial:
\begin{hscode}\SaveRestoreHook
\column{B}{@{}>{\hspre}l<{\hspost}@{}}%
\column{E}{@{}>{\hspre}l<{\hspost}@{}}%
\>[B]{}h_0 \mathrel{=}\Varid{fromList}\;\mathrm{3}\;\mathrm{1}\;[\mskip1.5mu \mathrm{1.0},\mathrm{1.0},\mathrm{1},\mathrm{0}\mskip1.5mu]{}\<[E]%
\ColumnHook
\end{hscode}\resethooks
\item \emph{Step function}:
\begin{hscode}\SaveRestoreHook
\column{B}{@{}>{\hspre}l<{\hspost}@{}}%
\column{E}{@{}>{\hspre}l<{\hspost}@{}}%
\>[B]{}\Varid{step}\;(\Varid{x},\Varid{h})\mathrel{=}(\alpha \;(\Varid{wy}\mathbin{*}\Varid{h}),\alpha \;(\Varid{wh}\mathbin{*}\Varid{h}\mathbin{+}\Varid{wx}\mathbin{*}\Varid{x})){}\<[E]%
\ColumnHook
\end{hscode}\resethooks
\item Função de activação:
\begin{hscode}\SaveRestoreHook
\column{B}{@{}>{\hspre}l<{\hspost}@{}}%
\column{E}{@{}>{\hspre}l<{\hspost}@{}}%
\>[B]{}\alpha \mathrel{=}\mathsf{fmap}\;\sigma \;\mathbf{where}\;\sigma \;\Varid{x}\mathrel{=}(\Varid{tanh}\;\Varid{x}\mathbin{+}\mathrm{1})\mathbin{/}\mathrm{2}{}\<[E]%
\ColumnHook
\end{hscode}\resethooks
\item \emph{Input layer}:
\begin{hscode}\SaveRestoreHook
\column{B}{@{}>{\hspre}l<{\hspost}@{}}%
\column{E}{@{}>{\hspre}l<{\hspost}@{}}%
\>[B]{}\Varid{inp}\mathrel{=}[\mskip1.5mu x_1 ,x_2 ,x_3 ,x_4 \mskip1.5mu]{}\<[E]%
\\
\>[B]{}x_1 \mathrel{=}\Varid{fromList}\;\mathrm{4}\;\mathrm{1}\;[\mskip1.5mu \mathrm{1.0},\mathrm{0},\mathrm{0},\mathrm{0}\mskip1.5mu]{}\<[E]%
\\
\>[B]{}x_2 \mathrel{=}\Varid{fromList}\;\mathrm{4}\;\mathrm{1}\;[\mskip1.5mu \mathrm{0},\mathrm{1.0},\mathrm{0},\mathrm{0}\mskip1.5mu]{}\<[E]%
\\
\>[B]{}x_3 \mathrel{=}\Varid{fromList}\;\mathrm{4}\;\mathrm{1}\;[\mskip1.5mu \mathrm{0},\mathrm{0},\mathrm{1.0},\mathrm{0}\mskip1.5mu]{}\<[E]%
\\
\>[B]{}x_4 \mathrel{=}x_3 {}\<[E]%
\ColumnHook
\end{hscode}\resethooks
\item Matrizes exemplo:
\begin{hscode}\SaveRestoreHook
\column{B}{@{}>{\hspre}l<{\hspost}@{}}%
\column{17}{@{}>{\hspre}l<{\hspost}@{}}%
\column{E}{@{}>{\hspre}l<{\hspost}@{}}%
\>[B]{}\Varid{wh}\mathrel{=}\Varid{fromList}\;\mathrm{3}\;\mathrm{3}\;[\mskip1.5mu \mathrm{0.4},\mathbin{-}\mathrm{0.2},\mathrm{1.6},\mathbin{-}\mathrm{3.1},\mathrm{1.4},\mathrm{0.1},\mathrm{5.4},\mathbin{-}\mathrm{2.7},\mathrm{0.1}\mskip1.5mu]{}\<[E]%
\\
\>[B]{}\Varid{wy}\mathrel{=}\Varid{fromList}\;\mathrm{4}\;\mathrm{3}\;[\mskip1.5mu \mathrm{2.1},\mathrm{1.1},\mathrm{0.8},\mathrm{1.3},\mathbin{-}\mathrm{6.4},\mathbin{-}\mathrm{3.4},\mathbin{-}\mathrm{2.7},\mathbin{-}\mathrm{3.8},\mathbin{-}\mathrm{1.3},\mathbin{-}\mathrm{0.5},\mathbin{-}\mathrm{0.9},\mathbin{-}\mathrm{0.4}\mskip1.5mu]{}\<[E]%
\\
\>[B]{}\Varid{wx}\mathrel{=}\Varid{fromLists}\;{}\<[17]%
\>[17]{}[\mskip1.5mu [\mskip1.5mu \mathrm{0.0},\mathbin{-}\mathrm{51.9},\mathrm{0.0},\mathrm{0.0}\mskip1.5mu],[\mskip1.5mu \mathrm{0.0},\mathrm{26.6},\mathrm{0.0},\mathrm{0.0}\mskip1.5mu],[\mskip1.5mu \mathbin{-}\mathrm{16.7},\mathbin{-}\mathrm{5.5},\mathbin{-}\mathrm{0.1},\mathrm{0.1}\mskip1.5mu]\mskip1.5mu]{}\<[E]%
\ColumnHook
\end{hscode}\resethooks
\end{itemize}
\textbf{NB}: Podem ser definidos e usados outros dados em função das experiências que se queiram fazer.

\subsection*{Problema 3}
Fórmula (\ref{eq:pi}) em Haskell:
\begin{hscode}\SaveRestoreHook
\column{B}{@{}>{\hspre}l<{\hspost}@{}}%
\column{6}{@{}>{\hspre}l<{\hspost}@{}}%
\column{E}{@{}>{\hspre}l<{\hspost}@{}}%
\>[B]{}\pi_{\mathit{calc}}\;\Varid{n}\mathrel{=}(\Varid{sum}\comp \map \;\Varid{f})\;[\mskip1.5mu \mathrm{0}\mathinner{\ldotp\ldotp}\Varid{n}\mskip1.5mu]\;\mathbf{where}{}\<[E]%
\\
\>[B]{}\hsindent{6}{}\<[6]%
\>[6]{}\Varid{f}\;\Varid{n}\mathrel{=}\Varid{fromIntegral}\;(\Varid{n} ! \mathbin{*}\Varid{n} ! \mathbin{*}(\Varid{g}\;\Varid{n}))\mathbin{/}\Varid{fromIntegral}\;(\Varid{d}\;\Varid{n}){}\<[E]%
\\
\>[B]{}\hsindent{6}{}\<[6]%
\>[6]{}\Varid{g}\;\Varid{n}\mathrel{=}\mathrm{2}\mathbin{\uparrow}(\Varid{n}\mathbin{+}\mathrm{1}){}\<[E]%
\\
\>[B]{}\hsindent{6}{}\<[6]%
\>[6]{}\Varid{d}\;\Varid{n}\mathrel{=}(\mathrm{2}\mathbin{*}\Varid{n}\mathbin{+}\mathrm{1}) ! {}\<[E]%
\ColumnHook
\end{hscode}\resethooks

\subsection*{Problema 4}
Se pedirmos ao \GHCi\ que nos mostre o vector \ensuremath{\Varid{vec}} obteremos:
\begin{tabbing}\ttfamily
~\char123{}~X1~\char124{}\char45{}\char62{}~2~\char44{}~X2~\char124{}\char45{}\char62{}~1~\char125{}\char13{}
\end{tabbing}
Este resultado aparece mediante a seguinte instância de \ensuremath{\Conid{Vec}} na classe \ensuremath{\Conid{Show}}:
\begin{hscode}\SaveRestoreHook
\column{B}{@{}>{\hspre}l<{\hspost}@{}}%
\column{5}{@{}>{\hspre}l<{\hspost}@{}}%
\column{8}{@{}>{\hspre}l<{\hspost}@{}}%
\column{18}{@{}>{\hspre}l<{\hspost}@{}}%
\column{33}{@{}>{\hspre}l<{\hspost}@{}}%
\column{37}{@{}>{\hspre}l<{\hspost}@{}}%
\column{E}{@{}>{\hspre}l<{\hspost}@{}}%
\>[B]{}\mathbf{instance}\;(\Conid{Show}\;\Varid{a},\Conid{Ord}\;\Varid{a},\Conid{Eq}\;\Varid{a})\Rightarrow \Conid{Show}\;(\Conid{Vec}\;\Varid{a})\;\mathbf{where}{}\<[E]%
\\
\>[B]{}\hsindent{5}{}\<[5]%
\>[5]{}\Varid{show}\mathrel{=}\Varid{showbag}\comp \Varid{consol}\comp \Varid{outV}\;{}\<[37]%
\>[37]{}\mathbf{where}{}\<[E]%
\\
\>[5]{}\hsindent{3}{}\<[8]%
\>[8]{}\Varid{showbag}\mathrel{=}\Varid{concat}\comp {}\<[E]%
\\
\>[8]{}\hsindent{10}{}\<[18]%
\>[18]{}(\mathbin{+\!\!+}[\mskip1.5mu \text{\ttfamily \char34 ~\char125 \char34}\mskip1.5mu])\comp {}\<[33]%
\>[33]{}(\text{\ttfamily \char34 \char123 ~\char34}\mathbin{:})\comp {}\<[E]%
\\
\>[8]{}\hsindent{10}{}\<[18]%
\>[18]{}(\Varid{intersperse}\;\text{\ttfamily \char34 ~,~\char34})\comp {}\<[E]%
\\
\>[8]{}\hsindent{10}{}\<[18]%
\>[18]{}\Varid{sort}\comp {}\<[E]%
\\
\>[8]{}\hsindent{10}{}\<[18]%
\>[18]{}(\map \;\Varid{f})\;\mathbf{where}\;\Varid{f}\;(\Varid{a},\Varid{b})\mathrel{=}(\Varid{show}\;\Varid{a})\mathbin{+\!\!+}\text{\ttfamily \char34 ~|->~\char34}\mathbin{+\!\!+}(\Varid{show}\;\Varid{b}){}\<[E]%
\ColumnHook
\end{hscode}\resethooks
Outros detalhes da implementação de \ensuremath{\Conid{Vec}} em Haskell:
\begin{hscode}\SaveRestoreHook
\column{B}{@{}>{\hspre}l<{\hspost}@{}}%
\column{4}{@{}>{\hspre}l<{\hspost}@{}}%
\column{5}{@{}>{\hspre}l<{\hspost}@{}}%
\column{12}{@{}>{\hspre}l<{\hspost}@{}}%
\column{18}{@{}>{\hspre}l<{\hspost}@{}}%
\column{E}{@{}>{\hspre}l<{\hspost}@{}}%
\>[B]{}\mathbf{instance}\;\Conid{Applicative}\;\Conid{Vec}\;\mathbf{where}{}\<[E]%
\\
\>[B]{}\hsindent{5}{}\<[5]%
\>[5]{}\Varid{pure}\mathrel{=}\Varid{return}{}\<[E]%
\\
\>[B]{}\hsindent{5}{}\<[5]%
\>[5]{}(\mathbin{<*>})\mathrel{=}\Varid{aap}{}\<[E]%
\\[\blanklineskip]%
\>[B]{}\mathbf{instance}\;(\Conid{Eq}\;\Varid{a})\Rightarrow \Conid{Eq}\;(\Conid{Vec}\;\Varid{a})\;\mathbf{where}{}\<[E]%
\\
\>[B]{}\hsindent{4}{}\<[4]%
\>[4]{}\Varid{b}\equiv \Varid{b'}\mathrel{=}(\Varid{outV}\;\Varid{b})\mathbin{`\Varid{lequal}`}(\Varid{outV}\;\Varid{b'}){}\<[E]%
\\
\>[4]{}\hsindent{8}{}\<[12]%
\>[12]{}\mathbf{where}\;\Varid{lequal}\;\Varid{a}\;\Varid{b}\mathrel{=}\Varid{isempty}\;(\Varid{a}\mathbin{\ominus}\Varid{b}){}\<[E]%
\\
\>[12]{}\hsindent{6}{}\<[18]%
\>[18]{}{\Varid{a}}\ominus{\Varid{b}}\mathrel{=}\Varid{a}\mathbin{+\!\!+}\overline{ \Varid{b}}{}\<[E]%
\\
\>[12]{}\hsindent{6}{}\<[18]%
\>[18]{}\overline{ \Varid{x}}\mathrel{=}[\mskip1.5mu (\Varid{k},\mathbin{-}\Varid{i})\mid (\Varid{k},\Varid{i})\leftarrow \Varid{x}\mskip1.5mu]{}\<[E]%
\ColumnHook
\end{hscode}\resethooks
Funções auxiliares:
\begin{hscode}\SaveRestoreHook
\column{B}{@{}>{\hspre}l<{\hspost}@{}}%
\column{E}{@{}>{\hspre}l<{\hspost}@{}}%
\>[B]{}\Varid{consol}\mathbin{::}(\Conid{Eq}\;\Varid{b})\Rightarrow [\mskip1.5mu (\Varid{b},\Conid{Int})\mskip1.5mu]\to [\mskip1.5mu (\Varid{b},\Conid{Int})\mskip1.5mu]{}\<[E]%
\\
\>[B]{}\Varid{consol}\mathrel{=}\Varid{filter}\;\Varid{nzero}\comp \map \;(\Varid{id}\times\Varid{sum})\comp \Varid{col}\;\mathbf{where}\;\Varid{nzero}\;(\anonymous ,\Varid{x})\mathrel{=}\Varid{x}\not\equiv \mathrm{0}{}\<[E]%
\\[\blanklineskip]%
\>[B]{}\Varid{isempty}\mathbin{::}\Conid{Eq}\;\Varid{a}\Rightarrow [\mskip1.5mu (\Varid{a},\Conid{Int})\mskip1.5mu]\to \Conid{Bool}{}\<[E]%
\\
\>[B]{}\Varid{isempty}\mathrel{=}\Varid{all}\;(\equiv \mathrm{0})\comp \map \;\p2\comp \Varid{consol}{}\<[E]%
\\[\blanklineskip]%
\>[B]{}\Varid{col}\mathbin{::}(\Conid{Eq}\;\Varid{a},\Conid{Eq}\;\Varid{b})\Rightarrow [\mskip1.5mu (\Varid{a},\Varid{b})\mskip1.5mu]\to [\mskip1.5mu (\Varid{a},[\mskip1.5mu \Varid{b}\mskip1.5mu])\mskip1.5mu]{}\<[E]%
\\
\>[B]{}\Varid{col}\;\Varid{x}\mathrel{=}\Varid{nub}\;[\mskip1.5mu \Varid{k}\mapsto[\mskip1.5mu \Varid{d'}\mid (\Varid{k'},\Varid{d'})\leftarrow \Varid{x},\Varid{k'}\equiv \Varid{k}\mskip1.5mu]\mid (\Varid{k},\Varid{d})\leftarrow \Varid{x}\mskip1.5mu]\;\mathbf{where}\;\Varid{a}\mapsto\Varid{b}\mathrel{=}(\Varid{a},\Varid{b}){}\<[E]%
\ColumnHook
\end{hscode}\resethooks




%----------------- Soluções dos alunos -----------------------------------------%

\section{Soluções dos alunos}\label{sec:resolucao}
Os alunos devem colocar neste anexo as suas soluções para os exercícios
propostos, de acordo com o ``layout'' que se fornece.
Não podem ser alterados os nomes ou tipos das funções dadas, mas pode ser
adicionado texto ao anexo, bem como diagramas e/ou outras funções auxiliares
que sejam necessárias.

\noindent
\textbf{Importante}: Não pode ser alterado o texto deste ficheiro fora deste anexo.

\subsection*{Problema 1}
O Problema 1 foi resolvido utilizando um \textit{hilomorfismo}, 
que consiste em um \textit{anamorfismo} que "faz pouco" e um 
\textit{catamorfismo} que "faz muito" — ou seja, uma abordagem do 
tipo \textit{Easy Split / Hard Join}.

Começamos com o \textit{anamorfismo}:


\begin{hscode}\SaveRestoreHook
\column{B}{@{}>{\hspre}l<{\hspost}@{}}%
\column{E}{@{}>{\hspre}l<{\hspost}@{}}%
\>[B]{}\Varid{mysuffixes}\mathbin{::}[\mskip1.5mu \Varid{a}\mskip1.5mu]\to [\mskip1.5mu [\mskip1.5mu \Varid{a}\mskip1.5mu]\mskip1.5mu]{}\<[E]%
\\
\>[B]{}\Varid{mysuffixes}\mathrel{=}\anaList{(\Varid{id}+\conj{\Varid{cons}}{\p2})\comp \Varid{outList}}{}\<[E]%
\ColumnHook
\end{hscode}\resethooks

De seguida, podemos começar pela cabeça de cada 
sublista e realizar uma busca exaustiva na lista de listas.


\begin{eqnarray*}
\xymatrix@C=2cm{
&
    \ensuremath{\N_0}^*
    \ar[r]^-{\ensuremath{\Varid{mysuffixes}}}
&
    (\ensuremath{\N_0}^*)^*
    \ar[r]_-{\ensuremath{\map \;\conj{\Varid{head}}{\Varid{tail}}}}
&
    \ensuremath{\N_0\times\N_0}^*
}
\end{eqnarray*}

No final, após realizar a procura e o cálculo em todas as sublistas, 
aplicamos o \textit{catamorfismo} \textit{mymaximum}, que extrai o maior
 elemento de uma lista.

\begin{hscode}\SaveRestoreHook
\column{B}{@{}>{\hspre}l<{\hspost}@{}}%
\column{E}{@{}>{\hspre}l<{\hspost}@{}}%
\>[B]{}\Varid{mymaximum}\mathbin{::}[\mskip1.5mu \Conid{Int}\mskip1.5mu]\to \Conid{Int}{}\<[E]%
\\
\>[B]{}\Varid{mymaximum}\mathrel{=}\llparenthesis\, \alt{\underline{\mathrm{0}}}{\uncurry{\Varid{max}}}\,\rrparenthesis{}\<[E]%
\ColumnHook
\end{hscode}\resethooks
Como de cada sublista calculada obtemos a área máxima, 
ficamos com uma lista de áreas máximas. Assim, aplicamos o cálculo do máximo sobre essa lista.
\begin{eqnarray*}
\xymatrix@C=2cm{
&
    \ensuremath{\N_0}^*
    \ar[r]^-{\ensuremath{\Varid{mymaximum}}}
&
    \ensuremath{\N_0}   
}
\end{eqnarray*} 
\textbf{Neste passo}, necessitamos de uma função \textit{area}
 que, a partir de um par \textit{head} e \textit{tail}, efectue o cálculo tendo 
 sempre em conta a cabeça.

\begin{hscode}\SaveRestoreHook
\column{B}{@{}>{\hspre}l<{\hspost}@{}}%
\column{E}{@{}>{\hspre}l<{\hspost}@{}}%
\>[B]{}\Varid{area}\mathbin{::}(\Conid{Int},[\mskip1.5mu \Conid{Int}\mskip1.5mu])\to \Conid{Int}{}\<[E]%
\ColumnHook
\end{hscode}\resethooks
\begin{eqnarray*}
\xymatrix@C=2cm{
    \ensuremath{\N_0\times\N_0}^*
    \ar[d]_-{\ensuremath{\Varid{id}\times((\map \;\Varid{swap})\comp \Varid{zip}\;[\mskip1.5mu \mathrm{1}\mathinner{\ldotp\ldotp}\mskip1.5mu])}}
    \ar[rr]^{\ensuremath{\Varid{area}}}
&
&
    \ensuremath{\N_0}
\\
    \ensuremath{\N_0\times(\N_0\times\N_0)}^*
    \ar[d]_-{\ensuremath{\conj{\Varid{id}\times\Varid{len}}{\p2}}}
&
& 
\\
    \ensuremath{(\N_0\times\N_0)\times(\N_0\times\N_0)}^*
    \ar[d]_-{\ensuremath{(\uncurry{\Varid{replicate}}\comp \Varid{swap})\times\Varid{id}}}
&
&  
\\
    \ensuremath{\N_0}^* \ensuremath{\times(\N_0\times\N_0)}^*
    \ar[r]_-{\ensuremath{\uncurry{\Varid{zip}}}}
&
    (\ensuremath{\N_0\times(\N_0\times\N_0)})^*
    \ar[r]_-{\ensuremath{\map \;\Varid{auxarea}}} 
&   
    \ensuremath{\N_0}^*
    \ar[uuu]_-{\ensuremath{\Varid{mymaximum}}} 
}
\end{eqnarray*}
Temos como função auxiliar a \textit{auxarea}, que efectua o cálculo e 
determina qual dos extremos é o menor para calcular a área.
\begin{hscode}\SaveRestoreHook
\column{B}{@{}>{\hspre}l<{\hspost}@{}}%
\column{E}{@{}>{\hspre}l<{\hspost}@{}}%
\>[B]{}\Varid{auxarea}\mathbin{::}(\Conid{Int},(\Conid{Int},\Conid{Int}))\to \Conid{Int}{}\<[E]%
\ColumnHook
\end{hscode}\resethooks

Recordo que, no par \ensuremath{\N_0\times(\N_0\times\N_0)}, a \textbf{primeira componente} é a 
\textit{head}, ou seja, uma das alturas que estamos a calcular; a \textbf{segunda componente} 
é a segunda altura; e, por fim, a \textbf{última componente} é a respetiva distância entre as 
duas alturas. Basta agora verificar qual das alturas é a menor e multiplicar pela distância.

\begin{eqnarray*}
\xymatrix@C=2cm{
    \ensuremath{\N_0\times(\N_0\times\N_0)}
    \ar[r]^-{\ensuremath{\Varid{assocl}}}
    \ar[d]_-{\ensuremath{\Varid{auxarea}}}
&
    \ensuremath{(\N_0\times\N_0)\times\N_0}  
    \ar[r]^-{\ensuremath{(\Varid{grd}\;\uncurry{\cdot }\;(\mathbin{>}))\times\Varid{id}}}
&
    \ensuremath{((\N_0\times\N_0)\mathbin{+}(\N_0\times\N_0))\times\N_0}
    \ar[ld]^-{\ensuremath{\alt{\p2}{\p1}\times\Varid{id}}}
\\
    \ensuremath{\N_0}
&
    \ensuremath{\N_0\times\N_0}
    \ar[l]^-{\ensuremath{\uncurry{(\mathbin{*})}}}
&  
}
\end{eqnarray*}
Podemos agora exprimir as funções \textit{area} e \textit{auxarea} em \Haskell\ da seguinte forma:
\begin{hscode}\SaveRestoreHook
\column{B}{@{}>{\hspre}l<{\hspost}@{}}%
\column{109}{@{}>{\hspre}l<{\hspost}@{}}%
\column{E}{@{}>{\hspre}l<{\hspost}@{}}%
\>[B]{}\Varid{auxarea}\mathrel{=}\uncurry{(\mathbin{*})}\comp (\alt{\p2}{\p1}\times\Varid{id})\comp (\Varid{grd}\;\uncurry{(\mathbin{>})}\times\Varid{id})\comp \Varid{assocl}{}\<[E]%
\\[\blanklineskip]%
\>[B]{}\Varid{area}\mathrel{=}\Varid{mymaximum}\comp \map \;\Varid{auxarea}\comp \uncurry{\Varid{zip}}\comp (\uncurry{\Varid{replicate}}\comp \Varid{swap}\times\Varid{id})\comp \conj{\Varid{id}\times\length }{\p2}{}\<[109]%
\>[109]{}\comp (\Varid{id}\times\map \;\Varid{swap}\comp \Varid{zip}\;[\mskip1.5mu \mathrm{1}\mathinner{\ldotp\ldotp}\mskip1.5mu]){}\<[E]%
\ColumnHook
\end{hscode}\resethooks
Finalmente, temos uma definição para o \textit{mostwater}, que pode ser representada no seguinte diagrama:
\begin{eqnarray*}
\xymatrix@C=2cm{
&
    \ensuremath{\N_0}^*
    \ar[d]_-{\ensuremath{\Varid{mysuffixes}}}
    \ar[rrdd]^-{\ensuremath{\Varid{mostwater}}}
\\
& 
    (\ensuremath{\N_0}^*)^*
    \ar[d]_-{\ensuremath{\map \;\conj{\Varid{head}}{\Varid{tail}}}}
\\
&
    (\ensuremath{\N_0\times\N_0}^*)^*
    \ar[r]_-{\ensuremath{\map \;\Varid{area}}}
&
    \ensuremath{\N_0}^*
    \ar[r]_-{\ensuremath{\Varid{mymaximum}}}
&
    \ensuremath{\N_0}
}
\end{eqnarray*}
Para transformar num hilomorfismo de listas, podemos utilizar algumas leis 
do \cp{Cálculo de Programas}, nomeadamente a 
\textit{Composição de maps} e a \textit{Absorção-Cata}.

\begin{eqnarray*}
\start
\ensuremath{\Varid{mostwater}\mathrel{=}\Varid{mymaximum}\comp \map \;\Varid{area}\comp \map \;\conj{\Varid{head}}{\Varid{tail}}\comp \Varid{mysuffixes}}
\just\equiv{Functor-F}
\ensuremath{\Varid{mostwater}\mathrel{=}\Varid{mymaximum}\comp \map \;(\Varid{area}\comp \conj{\Varid{head}}{\Varid{tail}})\comp \Varid{mysuffixes}}
\just\equiv{mymaximum-definition}
\ensuremath{\Varid{mostwater}\mathrel{=}\llparenthesis\, \alt{\underline{\mathrm{0}}}{\uncurry{\Varid{max}}}\,\rrparenthesis\comp \map \;(\Varid{area}\comp \conj{\Varid{head}}{\Varid{tail}})\comp \Varid{mysuffixes}}
\just\equiv{Absorção-cata}
\ensuremath{\Varid{mostwater}\mathrel{=}\llparenthesis\, \alt{\underline{\mathrm{0}}}{\uncurry{\Varid{max}}}\comp (\mathrm{1}\mathbin{+}((\Varid{area}\comp \conj{\Varid{head}}{\Varid{tail}})\times\Varid{id}))\,\rrparenthesis\comp \Varid{mysuffixes}}
\just\equiv{Absorção-+, mysuffixes-definition}
\ensuremath{\Varid{mostwater}\mathrel{=}\llparenthesis\, \alt{\underline{\mathrm{0}}}{\uncurry{\Varid{max}}\comp ((\Varid{area}\comp \conj{\Varid{head}}{\Varid{tail}})\times\Varid{id})}\,\rrparenthesis\comp \anaList{(\Varid{id}\mathbin{+}\conj{\Varid{cons}}{\p2})\comp \Varid{outList}}}
\qed
\end{eqnarray*}
Temos a definição de \textit{mostwater} como um \textit{hilomorfismo}:
\begin{hscode}\SaveRestoreHook
\column{B}{@{}>{\hspre}l<{\hspost}@{}}%
\column{91}{@{}>{\hspre}l<{\hspost}@{}}%
\column{E}{@{}>{\hspre}l<{\hspost}@{}}%
\>[B]{}\Varid{mostwater}\mathrel{=}\Varid{hyloList}\;\alt{\underline{\mathrm{0}}}{\uncurry{\Varid{max}}\comp ((\Varid{area}\comp \conj{\Varid{head}}{\Varid{tail}})\times\Varid{id})}\;{}\<[91]%
\>[91]{}((\Varid{id}+\conj{\Varid{cons}}{\p2})\comp \Varid{outList}){}\<[E]%
\ColumnHook
\end{hscode}\resethooks
Podemos agora exprimir o \textit{mostwater} no seu diagrama de \textit{hilomorfismo}, 
evidenciando o seu \textit{divide} e o seu \textit{conquer}, bem como explicitando a sua 
estrutura intermédia:

\begin{eqnarray*}
\xymatrix@C=2cm{
    \ensuremath{\N_0}^*
    \ar[r]^-{\ensuremath{\Varid{out}}_*}
    \ar[d]_-{\ensuremath{\Varid{divide}}}
&
    \ensuremath{\mathrm{1}\mathbin{+}\N_0\times\N_0}^*
    \ar[r]^-{\ensuremath{(\Varid{id}\mathbin{+}\conj{\Varid{cons}}{\p2})}}
&
    \ensuremath{\mathrm{1}\mathbin{+}\N_0}^* \ensuremath{\times\N_0}^*
    \ar[d]^-{\ensuremath{\Varid{id}\mathbin{+}\Varid{id}\times\Varid{divide}}}
\\
    (\ensuremath{\N_0}^*)^*
    \ar[d]_-{conquer}
    \ar@/^/[rr]^-{\ensuremath{\Varid{out}}_{**}}
&
&
    \ensuremath{\mathrm{1}\mathbin{+}\N_0}^* \ensuremath{\times} ( \ensuremath{\N_0}^*)^*
    \ar@/^/[ll]^-{in_{**}}
    \ar[d]^{\ensuremath{\Varid{id}\mathbin{+}\Varid{id}\times\Varid{conquer}}}
\\
    \ensuremath{\N_0}
&
&
    \ensuremath{\mathrm{1}\mathbin{+}\N_0}^* \ensuremath{\times\N_0}
    \ar[ll]^-{\ensuremath{\alt{\underline{\mathrm{0}}}{\uncurry{\Varid{max}}\comp ((\Varid{area}\comp \conj{\Varid{head}}{\Varid{tail}})\times\Varid{id})}}} 
}
\end{eqnarray*}



\subsection*{Problema 2}

Para resolver este problema, foi criada uma estrutura que facilita a visualização da solução.


\begin{eqnarray*}
\xymatrix@C=2cm{
    \ensuremath{\Conid{A}}^* \ensuremath{\times\Conid{S}}
    \ar@/^/[r]^-{\ensuremath{\Varid{outLP}}}
&
   \ensuremath{\mathrm{1}\times\Conid{S}} + \ensuremath{\Conid{A}}^* \ensuremath{\times(\Conid{A}}^* \ensuremath{\times\Conid{S})}
    \ar@/^/[l]^-{\ensuremath{\Varid{inLP}}}
}
\end{eqnarray*}
Tal que o \textit{inLP} é definido como a construção
 de uma lista, mantendo o valor da segunda componente:

\begin{hscode}\SaveRestoreHook
\column{B}{@{}>{\hspre}l<{\hspost}@{}}%
\column{E}{@{}>{\hspre}l<{\hspost}@{}}%
\>[B]{}\Varid{inLP}\mathrel{=}\alt{\Varid{nil}\times\Varid{id}}{(\uncurry{(\mathbin{+\!\!+})}\times\Varid{id})\comp \Varid{assocl}}{}\<[E]%
\ColumnHook
\end{hscode}\resethooks
O \textit{outLP} é definido como uma alternativa entre a primeira componente 
ser a lista vazia ou não, colocando assim o único valor numa lista.

\begin{hscode}\SaveRestoreHook
\column{B}{@{}>{\hspre}l<{\hspost}@{}}%
\column{E}{@{}>{\hspre}l<{\hspost}@{}}%
\>[B]{}\Varid{outLP}\;([\mskip1.5mu \mskip1.5mu],\Varid{s})\mathrel{=}i_1\;((),\Varid{s}){}\<[E]%
\\
\>[B]{}\Varid{outLP}\;(\Varid{h}\mathbin{:}\Varid{t},\Varid{s})\mathrel{=}i_2\;(\Varid{singl}\;\Varid{h},(\Varid{t},\Varid{s})){}\<[E]%
\ColumnHook
\end{hscode}\resethooks

Com o \textit{inLP} e o \textit{outLP} definidos, 
podemos definir o seu functor recursivo, o que nos 
permite admitir o seu \textit{catamorfismo} e o seu \textit{anamorfismo}.

Começamos pelo \textit{catamorfismo}, 
concretamente o \textit{catamorfismo} identidade, sendo o seu gerador o \textit{inLP}:

\begin{eqnarray*}
\xymatrix@C=2cm{
    \ensuremath{\Conid{A}}^* \ensuremath{\times\Conid{S}}
    \ar[r]^-{\ensuremath{\Varid{outLP}}}
    \ar[d]_-{\ensuremath{\Varid{cataid}}}
&
    \ensuremath{\mathrm{1}\times\Conid{S}} + \ensuremath{\Conid{A}}^* \ensuremath{\times(\Conid{A}}^* \ensuremath{\times\Conid{S})}
    \ar[d]^-{\ensuremath{\Varid{id}\mathbin{+}(\Varid{id}\times\Varid{cataid})}}
\\
    \ensuremath{\Conid{A}}^* \ensuremath{\times\Conid{S}}
&
    \ensuremath{\mathrm{1}\times\Conid{S}} + \ensuremath{\Conid{A}}^* \ensuremath{\times(\Conid{A}}^* \ensuremath{\times\Conid{S})}
    \ar[l]^-{\ensuremath{\Varid{inLP}}}
}
\end{eqnarray*}

A correção pode ser confirmada pela análise dos tipos, sabendo-se que esta é análoga ao \textit{anamorfismo}. 
O presente diagrama possibilita a visualização clara do funtor recursivo da estrutura \textbf{LP}, 
definido como:

\begin{hscode}\SaveRestoreHook
\column{B}{@{}>{\hspre}l<{\hspost}@{}}%
\column{E}{@{}>{\hspre}l<{\hspost}@{}}%
\>[B]{}\Varid{recLP}\;\Varid{f}\mathrel{=}\Varid{id}+(\Varid{id}\times\Varid{f}){}\<[E]%
\ColumnHook
\end{hscode}\resethooks
Com o \textit{recLP}, o \textit{inLP} e o \textit{outLP} ja podemos admitir catamorfismo
e anamorfismos
\begin{hscode}\SaveRestoreHook
\column{B}{@{}>{\hspre}l<{\hspost}@{}}%
\column{E}{@{}>{\hspre}l<{\hspost}@{}}%
\>[B]{}\Varid{cataLP}\;\Varid{g}\mathrel{=}\Varid{g}\comp \Varid{recLP}\;(\Varid{cataLP}\;\Varid{g})\comp \Varid{outLP}{}\<[E]%
\\[\blanklineskip]%
\>[B]{}\Varid{anaLP}\;\Varid{g}\mathrel{=}\Varid{inLP}\comp \Varid{recLP}\;(\Varid{anaLP}\;\Varid{g})\comp \Varid{g}{}\<[E]%
\ColumnHook
\end{hscode}\resethooks

Agora, pela ordem natural de \textbf{computação} dos \textit{catamorfismos} e \textit{anamorfismos}, 
podemos definir as funções \textit{mapAccumRfilter} e \textit{mapAccumLfilter} da seguinte forma:

\begin{hscode}\SaveRestoreHook
\column{B}{@{}>{\hspre}l<{\hspost}@{}}%
\column{E}{@{}>{\hspre}l<{\hspost}@{}}%
\>[B]{}\Varid{mapAccumRfilter}\;\Varid{h}\;\Varid{f}\mathrel{=}\Varid{cataLP}\;\alt{\Varid{nil}\times\Varid{id}}{\Varid{auxR}\;\Varid{h}\;\Varid{f}}{}\<[E]%
\\[\blanklineskip]%
\>[B]{}\Varid{mapAccumLfilter}\;\Varid{h}\;\Varid{f}\mathrel{=}\Varid{anaLP}\;((\Varid{id}+\Varid{auxL}\;\Varid{h}\;\Varid{f})\comp \Varid{outLP}){}\<[E]%
\ColumnHook
\end{hscode}\resethooks

Faltam ainda alguns passos de \textbf{cálculo}
para determinar as funções \textit{auxR} e \textit{auxL}. 
Antes disso, apresentam-se os diagramas que explicitam os tipos de ambas as funções.


\begin{eqnarray*}
\xymatrix@C=2cm{
    \ensuremath{\Conid{A}}^* \ensuremath{\times\Conid{S}}
    \ar[r]^-{\ensuremath{\Varid{outLP}}}
    \ar[d]_-{\ensuremath{\Varid{mapAccumRfilter}\;\Varid{h}\;\Varid{f}}}
&
    \ensuremath{\mathrm{1}\times\Conid{S}} + \ensuremath{\Conid{A}}^* \ensuremath{\times(\Conid{A}}^* \ensuremath{\times\Conid{S})}
    \ar[d]^-{\ensuremath{\Varid{id}\mathbin{+}(\Varid{id}\times\Varid{mapAccumRfilter}\;\Varid{h}\;\Varid{f})}}
\\
    \ensuremath{\Conid{B}}^* \ensuremath{\times\Conid{S}}
&
    \ensuremath{\mathrm{1}\times\Conid{S}} + \ensuremath{\Conid{A}}^* \ensuremath{\times(\Conid{B}}^* \ensuremath{\times\Conid{S})}
    \ar[l]^-{\ensuremath{\alt{\Varid{nil}\times\Varid{id}}{\Varid{auxR}\;\Varid{h}\;\Varid{f}}}}
}
\end{eqnarray*}

tipo de auxR:
\begin{eqnarray*}
\xymatrix@C=2cm{
    \ensuremath{\Conid{A}}^* \ensuremath{\times(\Conid{B}}^* \ensuremath{\times\Conid{S})}
    \ar[r]^-{\ensuremath{\Varid{auxR}\;\Varid{h}\;\Varid{f}}}
&
    \ensuremath{\Conid{B}}^* \ensuremath{\times\Conid{S}}
}
\end{eqnarray*}
ou em \Haskell:
\begin{hscode}\SaveRestoreHook
\column{B}{@{}>{\hspre}l<{\hspost}@{}}%
\column{E}{@{}>{\hspre}l<{\hspost}@{}}%
\>[B]{}\Varid{auxR}\mathbin{::}((\Varid{a},\Varid{s})\to \Conid{Bool})\to ((\Varid{a},\Varid{s})\to (\Varid{c},\Varid{s}))\to ([\mskip1.5mu \Varid{a}\mskip1.5mu],([\mskip1.5mu \Varid{c}\mskip1.5mu],\Varid{s}))\to ([\mskip1.5mu \Varid{c}\mskip1.5mu],\Varid{s}){}\<[E]%
\ColumnHook
\end{hscode}\resethooks

Segue-se o desenvolvimento do diagrama a partir do tipo inicial, 
tendo em conta as funções \textit{h} (verificação) e \textit{f} (transformação).

\begin{eqnarray*}
\xymatrix@C=2cm{
    \ensuremath{\Conid{A}}^* \ensuremath{\times(\Conid{B}}^* \ensuremath{\times\Conid{S})}
    \ar[r]^-{\ensuremath{\Varid{head}\times\Varid{id}}}
&
    \ensuremath{\Conid{A}\times(\Conid{B}}^* \ensuremath{\times\Conid{S})}
    \ar[d]_-{\ensuremath{\conj{\Varid{id}\times\p2}{\Varid{assocl}\comp (\Varid{id}\times\Varid{swap})}}}
\\
&
    \ensuremath{(\Conid{A}\times\Conid{S})\times((\Conid{A}\times\Conid{S})\times\Conid{B}}^*)
    \ar[d]_-{\ensuremath{(\Varid{h}\times(\Varid{f}\times\Varid{id}))}}
\\
&
    \ensuremath{\mathrm{2}\times((\Conid{B}\times\Conid{S})\times\Conid{B}}^*)
    \ar[d]_-{\ensuremath{\Varid{grd}\;\p1}}
\\
&
    \ensuremath{\mathrm{2}\times((\Conid{B}\times\Conid{S})\times\Conid{B}}^*) + \ensuremath{\mathrm{2}\times((\Conid{B}\times\Conid{S})\times\Conid{B}}^*)
    \ar[ld]_-{\ensuremath{\p2}}
    \ar[rd]^-{\ensuremath{\p2}}
    \ar[dd]_-{\ensuremath{\Varid{choice}}}
\\
    \ensuremath{((\Conid{B}\times\Conid{S})\times\Conid{B}}^*)
    \ar[rd]_-{\ensuremath{\Varid{swap}\comp (\Varid{id}\times\uncurry{(\mathbin{:})})\comp \Varid{assocr}\comp (\Varid{swap}\times\Varid{id})}}
&
&
    \ensuremath{((\Conid{B}\times\Conid{S})\times\Conid{B}}^*)
    \ar[ld]^-{\ensuremath{\Varid{swap}\comp (\p2\times\Varid{id})}}
\\
&
    \ensuremath{\Conid{B}}^* \ensuremath{\times\Conid{S}}
&
}
\end{eqnarray*}
Apresenta-se a definição de \textit{auxR}:
\begin{hscode}\SaveRestoreHook
\column{B}{@{}>{\hspre}l<{\hspost}@{}}%
\column{5}{@{}>{\hspre}l<{\hspost}@{}}%
\column{13}{@{}>{\hspre}l<{\hspost}@{}}%
\column{E}{@{}>{\hspre}l<{\hspost}@{}}%
\>[B]{}\Varid{auxR}\;\Varid{h}\;\Varid{f}\mathrel{=}{}\<[13]%
\>[13]{}\Varid{choice}\comp \Varid{grd}\;\p1\comp (\Varid{h}\times(\Varid{f}\times\Varid{id}))\comp \conj{\Varid{id}\times\p2}{\Varid{assocl}\comp (\Varid{id}\times\Varid{swap})}\comp (\Varid{head}\times\Varid{id})\;\mathbf{where}{}\<[E]%
\\
\>[B]{}\hsindent{5}{}\<[5]%
\>[5]{}\Varid{choice}\mathrel{=}\alt{\Varid{swap}\comp (\Varid{id}\times\uncurry{(\mathbin{:})})\comp \Varid{assocr}\comp (\Varid{swap}\times\Varid{id})\comp \p2}{\Varid{swap}\comp (\p2\times\Varid{id})\comp \p2}{}\<[E]%
\ColumnHook
\end{hscode}\resethooks
Segue-se uma definição análoga para \textit{auxL}, partindo de um \textit{anamorfismo}:

\begin{eqnarray*}
\xymatrix@C=2cm{
    \ensuremath{\Conid{B}}^* \ensuremath{\times\Conid{S}}
&
&
    \ensuremath{\mathrm{1}\times\Conid{S}\mathbin{+}\Conid{B}}^* \ensuremath{\times(\Conid{B}}^* \ensuremath{\times\Conid{S})}
    \ar[ll]_-{\ensuremath{\Varid{inLP}}}
\\
\ensuremath{\Conid{A}}^* \ensuremath{\times\Conid{S}}
\ar[u]^-{\ensuremath{\Varid{mapAccumLfilter}}}
\ar[r]_-{\ensuremath{\Varid{outLP}}}
&
    \ensuremath{\mathrm{1}\times\Conid{S}} + \ensuremath{\Conid{A}}^* \ensuremath{\times(\Conid{A}}^* \ensuremath{\times\Conid{S})}
    \ar[r]_-{\ensuremath{\Varid{id}\mathbin{+}\Varid{auxL}\;\Varid{h}\;\Varid{f}}}
&
    \ensuremath{\mathrm{1}\times\Conid{S}} + \ensuremath{\Conid{C}}^* \ensuremath{\times(\Conid{A}}^* \ensuremath{\times\Conid{S})}
    \ar[u]_-{\ensuremath{\Varid{is}\mathbin{+}(\Varid{id}\times\Varid{mapAccumLfilter})}}
}
\end{eqnarray*}

Considera-se, então, o tipo da função \textit{auxL}:

\begin{eqnarray*}
\xymatrix@C=2cm{
&
    \ensuremath{\Conid{A}}^* \ensuremath{\times(\Conid{A}}^* \ensuremath{\times\Conid{S})}
    \ar[r]_-{\ensuremath{\Varid{auxL}\;\Varid{h}\;\Varid{f}}}
&
    \ensuremath{\Conid{C}}^* \ensuremath{\times(\Conid{A}}^* \ensuremath{\times\Conid{S})}
}
\end{eqnarray*}
Definido em \Haskell\ como:

\begin{hscode}\SaveRestoreHook
\column{B}{@{}>{\hspre}l<{\hspost}@{}}%
\column{E}{@{}>{\hspre}l<{\hspost}@{}}%
\>[B]{}\Varid{auxL}\mathbin{::}((\Varid{a},\Varid{s})\to \Conid{Bool})\to ((\Varid{a},\Varid{s})\to (\Varid{c},\Varid{s}))\to ([\mskip1.5mu \Varid{a}\mskip1.5mu],([\mskip1.5mu \Varid{a}\mskip1.5mu],\Varid{s}))\to ([\mskip1.5mu \Varid{c}\mskip1.5mu],([\mskip1.5mu \Varid{a}\mskip1.5mu],\Varid{s})){}\<[E]%
\ColumnHook
\end{hscode}\resethooks

Segue-se o desenvolvimento do diagrama a partir do tipo inicial, tendo em conta as 
funções \textit{h} (verificação) e \textit{f} (transformação), com o objetivo de 
alcançar uma conclusão para o código da função \textit{auxL}

\begin{eqnarray*}
\xymatrix@C=2cm{
    \ensuremath{\Conid{A}\times(\Conid{S}\times} \ensuremath{\Conid{A}}^* \ensuremath{)}
        \ar[d]^-{\ensuremath{\Varid{assocl}}}
    &
    \ensuremath{\Conid{A}}^* \ensuremath{\times(} \ensuremath{\Conid{A}}^* \ensuremath{\times\Conid{S})}
        \ar[l]^-{\ensuremath{(\Varid{head}\times\Varid{swap})}}
        \ar[rdddd]^-{\ensuremath{\Varid{auxL}\;\Varid{h}\;\Varid{f}}}
\\ 
    \ensuremath{(\Conid{A}\times\Conid{S})\times} \ensuremath{\Conid{A}}^*
        \ar[d]^-{\ensuremath{(\conj{\Varid{h}}{\Varid{f}}\times\Varid{id})}}
\\
    \ensuremath{\mathrm{2}\times(\Conid{C}\times\Conid{S})\times} \ensuremath{\Conid{A}}^*
        \ar[d]^-{\ensuremath{\Varid{assocr}}}
\\ 
    \ensuremath{\mathrm{2}\times((\Conid{C}\times\Conid{S})\times} \ensuremath{\Conid{A}}^* \ensuremath{)}
        \ar[d]^-{\ensuremath{(\Varid{id}\times((\Varid{id}\times\Varid{swap})\comp \Varid{assocr}))}}
\\
    \ensuremath{\mathrm{2}\times(\Conid{C}\times(} \ensuremath{\Conid{A}}^* \ensuremath{\times\Conid{S}))}
        \ar[rr]^-{\ensuremath{\Varid{choice}}}
        \ar[d]^-{\ensuremath{\Varid{grd}\;\p1}}
&
&
    \ensuremath{\Conid{C}}^* \ensuremath{\times(} \ensuremath{\Conid{A}}^* \ensuremath{\times\Conid{S})}
\\
    \ensuremath{\mathrm{2}\times((\Conid{C}\times\Conid{S})\times} \ensuremath{\Conid{A}}^* \ensuremath{)\mathbin{+}\mathrm{2}\times((\Conid{C}\times\Conid{S})\times} \ensuremath{\Conid{A}}^* \ensuremath{)}
        \ar[d]^-{\ensuremath{\p2\mathbin{+}\p2}}
\\
    \ensuremath{(\Conid{C}\times\Conid{S})\times} \ensuremath{\Conid{A}}^*
        \ar[d]^-{\ensuremath{\Varid{assocr}\mathbin{+}\Varid{assocr}}}
\\ 
    \ensuremath{\Conid{C}\times(\Conid{S}\times} \ensuremath{\Conid{A}}^* \ensuremath{)\mathbin{+}\Conid{C}\times(\Conid{S}\times} \ensuremath{\Conid{A}}^* \ensuremath{)}
        \ar[d]^-{\ensuremath{(\Varid{singl}\times\Varid{swap})\mathbin{+}(\Varid{nil}\times\Varid{swap})}}
\\ 
    \ensuremath{\Conid{C}}^* \ensuremath{\times(} \ensuremath{\Conid{A}}^* \ensuremath{\times\Conid{S})\mathbin{+}} \ensuremath{\Conid{C}}^* \ensuremath{\times(} \ensuremath{\Conid{A}}^* \ensuremath{\times\Conid{S})}
        \ar[r]^-{\ensuremath{\alt{\Varid{id}}{\Varid{id}}}}
&
    \ensuremath{\Conid{C}}^* \ensuremath{\times(} \ensuremath{\Conid{A}}^* \ensuremath{\times\Conid{S})}
     \ar[ruuuu]^-{\ensuremath{\Varid{id}}}
}
\end{eqnarray*}
Desta forma, obtém-se finalmente a definição de \textit{auxL} em \Haskell:
\begin{hscode}\SaveRestoreHook
\column{B}{@{}>{\hspre}l<{\hspost}@{}}%
\column{5}{@{}>{\hspre}l<{\hspost}@{}}%
\column{13}{@{}>{\hspre}l<{\hspost}@{}}%
\column{21}{@{}>{\hspre}l<{\hspost}@{}}%
\column{E}{@{}>{\hspre}l<{\hspost}@{}}%
\>[B]{}\Varid{auxL}\;\Varid{h}\;\Varid{f}\mathrel{=}{}\<[13]%
\>[13]{}\Varid{choice}{}\<[21]%
\>[21]{}\comp (\Varid{id}\times((\Varid{id}\times\Varid{swap})\comp \Varid{assocr}))\comp \Varid{assocr}\comp (\conj{\Varid{h}}{\Varid{f}}\times\Varid{id})\comp \Varid{assocl}\comp (\Varid{head}\times\Varid{swap})\;\mathbf{where}{}\<[E]%
\\
\>[B]{}\hsindent{5}{}\<[5]%
\>[5]{}\Varid{choice}\mathrel{=}\alt{(\Varid{singl}\times\Varid{id})\comp \p2}{(\Varid{nil}\times\Varid{id})\comp \p2}\comp \Varid{grd}\;\p1{}\<[E]%
\ColumnHook
\end{hscode}\resethooks

\subsection*{Problema 3}
O Problema 3 foi resolvido por recursividade mútua, tendo como objetivo 
definir as funções que realizam chamadas recursivas mútuas entre si. Para tal, 
é necessária a forma recursiva do somatório em \Haskell.

\begin{hscode}\SaveRestoreHook
\column{B}{@{}>{\hspre}l<{\hspost}@{}}%
\column{E}{@{}>{\hspre}l<{\hspost}@{}}%
\>[B]{}\Varid{picalcRec}\mathbin{::}\Conid{Fractional}\;\Varid{a}\Rightarrow \mathbb{Z}\to \Varid{a}{}\<[E]%
\\
\>[B]{}\Varid{picalcRec}\;\mathrm{0}\mathrel{=}\mathrm{2}{}\<[E]%
\\
\>[B]{}\Varid{picalcRec}\;\Varid{n}\mathrel{=}\Varid{fromIntegral}\;(\Varid{\Conid{Nat}.fac}\;\Varid{n})\mathbin{\uparrow}\mathrm{2}\mathbin{*}\mathrm{2}\mathbin{\uparrow}(\Varid{n}\mathbin{+}\mathrm{1})\mathbin{/}\Varid{fromIntegral}\;(\Varid{\Conid{Nat}.fac}\;(\mathrm{2}\mathbin{*}\Varid{n}\mathbin{+}\mathrm{1}))\mathbin{+}\pi_{\mathit{calc}}\;(\Varid{n}\mathbin{-}\mathrm{1}){}\<[E]%
\ColumnHook
\end{hscode}\resethooks

Como nesta definição existem várias dependências de \(n\), surge a 
necessidade de aplicar a \textit{regra da algibeira}. Assim, para cada 
ocorrência de \(n\), devemos associar uma função qualquer \(f(n-1)\).

\begin{eqnarray*}
\start
\ensuremath{\Varid{n}\mathbin{!\char94 }\mathrm{2}\mathbin{*}\mathrm{2}\mathbin{\uparrow}\{\mskip1.5mu \Varid{n}\mathbin{+}\mathrm{1}\mskip1.5mu\}\mathbin{/}(\mathrm{2}\mathbin{*}\Varid{n}\mathbin{+}\mathrm{1})\mathbin{!}\mathbin{+}\pi_{\mathit{calc}}\;(\Varid{n}\mathbin{-}\mathrm{1})}
\just\equiv{\ensuremath{\mathbf{let}\;\Varid{f}\;(\Varid{n}\mathbin{-}\mathrm{1})\mathrel{=}\Varid{n}\mathbin{!}}}
\ensuremath{\Varid{f}\;(\Varid{n}\mathbin{-}\mathrm{1})\mathrel{=}\Varid{n}\mathbin{!}}
    \Rightarrow
\ensuremath{\Varid{f}\;\Varid{n}\mathrel{=}(\Varid{n}\mathbin{+}\mathrm{1})\mathbin{!}}
    \Rightarrow
\ensuremath{\begin{lcbr}\Varid{f}\;\mathrm{0}\mathrel{=}\mathrm{1}\\\Varid{f}\;\Varid{n}\mathrel{=}(\Varid{n}\mathbin{+}\mathrm{1})\mathbin{*}\Varid{f}\;(\Varid{n}\mathbin{-}\mathrm{1})\end{lcbr}}
\just\equiv{\ensuremath{\mathbf{let}\;\Varid{f2}\;(\Varid{n}\mathbin{-}\mathrm{1})\mathrel{=}\Varid{n}\mathbin{+}\mathrm{1}}}
\ensuremath{\begin{lcbr}\begin{lcbr}\Varid{f}\;\mathrm{0}\mathrel{=}\mathrm{1}\\\Varid{f}\;\Varid{n}\mathrel{=}\Varid{f2}\;(\Varid{n}\mathbin{-}\mathrm{1})\mathbin{*}\Varid{f}\;(\Varid{n}\mathbin{-}\mathrm{1})\end{lcbr}\\\begin{lcbr}\Varid{f2}\;\mathrm{0}\mathrel{=}\mathrm{2}\\\Varid{f2}\;\Varid{n}\mathrel{=}\Varid{n}\mathbin{+}\mathrm{2}\end{lcbr}\end{lcbr}}
\end{eqnarray*}

Feito o procedimento para o \(n!\), aplicamos a mesma regra a
 todas as ocorrências de \(n\) e, por fim, extraímos os seus valores iniciais 
 e operações para o \textit{ciclo for}.

\begin{eqnarray*}
\start
\ensuremath{\Varid{f}\;(\Varid{n}\mathbin{-}\mathrm{1})\mathbin{\uparrow}\mathrm{2}\mathbin{*}\mathrm{2}\mathbin{\uparrow}\{\mskip1.5mu \Varid{n}\mathbin{+}\mathrm{1}\mskip1.5mu\}\mathbin{/}(\mathrm{2}\mathbin{*}\Varid{n}\mathbin{+}\mathrm{1})\mathbin{!}\mathbin{+}\Varid{picalcRec}\;(\Varid{n}\mathbin{-}\mathrm{1})}
\just\equiv{\ensuremath{\mathbf{let}\;\Varid{t}\;(\Varid{n}\mathbin{-}\mathrm{1})\mathrel{=}\mathrm{2}\mathbin{\uparrow}\{\mskip1.5mu \Varid{n}\mathbin{+}\mathrm{1}\mskip1.5mu\}}}
\ensuremath{\Varid{t}\;(\Varid{n}\mathbin{-}\mathrm{1})\mathrel{=}\mathrm{2}\mathbin{\uparrow}\{\mskip1.5mu \Varid{n}\mathbin{+}\mathrm{1}\mskip1.5mu\}}
    \Rightarrow
\ensuremath{\Varid{t}\;\Varid{n}\mathrel{=}\mathrm{2}\mathbin{\uparrow}\{\mskip1.5mu \Varid{n}\mathbin{+}\mathrm{2}\mskip1.5mu\}}
    \Rightarrow
\ensuremath{\begin{lcbr}\Varid{t}\;\mathrm{0}\mathrel{=}\mathrm{4}\\\Varid{t}\;\Varid{n}\mathrel{=}\mathrm{2}\mathbin{*}\Varid{t}\;(\Varid{n}\mathbin{-}\mathrm{1})\end{lcbr}}
\end{eqnarray*}

\begin{eqnarray*}
\start
\ensuremath{\Varid{f}\;(\Varid{n}\mathbin{-}\mathrm{1})\mathbin{\uparrow}\mathrm{2}\mathbin{*}\Varid{t}\;(\Varid{n}\mathbin{-}\mathrm{1})\mathbin{/}(\mathrm{2}\mathbin{*}\Varid{n}\mathbin{+}\mathrm{1})\mathbin{!}\mathbin{+}\pi_{\mathit{calc}}\;(\Varid{n}\mathbin{-}\mathrm{1})}
\just\equiv{\ensuremath{\mathbf{let}\;\Varid{g}\;(\Varid{n}\mathbin{-}\mathrm{1})\mathrel{=}(\mathrm{2}\mathbin{*}\Varid{n}\mathbin{+}\mathrm{1})\mathbin{!}}}
\ensuremath{\Varid{g}\;(\Varid{n}\mathbin{-}\mathrm{1})\mathrel{=}(\mathrm{2}\mathbin{*}\Varid{n}\mathbin{+}\mathrm{1})\mathbin{!}}
    \Rightarrow
\ensuremath{\Varid{g}\;\Varid{n}\mathrel{=}(\mathrm{2}\mathbin{*}\Varid{n}\mathbin{+}\mathrm{3})\mathbin{!}}
    \Rightarrow
\ensuremath{\begin{lcbr}\Varid{g}\;\mathrm{0}\mathrel{=}\mathrm{6}\\\Varid{g}\;\Varid{n}\mathrel{=}(\mathrm{2}\mathbin{*}\Varid{n}\mathbin{+}\mathrm{2})\mathbin{*}(\mathrm{2}\mathbin{*}\Varid{n}\mathbin{+}\mathrm{3})\mathbin{*}\Varid{g}\;(\Varid{n}\mathbin{-}\mathrm{1})\end{lcbr}}
\just\equiv{\ensuremath{\mathbf{let}\;\Varid{g1}\;(\Varid{n}\mathbin{-}\mathrm{1})\mathrel{=}\mathrm{2}\mathbin{*}\Varid{n}\mathbin{+}\mathrm{2}\;\Varid{and}\;\Varid{g2}\;(\Varid{n}\mathbin{-}\mathrm{1})\mathrel{=}(\mathrm{2}\mathbin{*}\Varid{n}\mathbin{+}\mathrm{3})}}
\ensuremath{\begin{lcbr}\begin{lcbr}\Varid{g1}\;\mathrm{0}\mathrel{=}\mathrm{4}\\\Varid{g1}\;\Varid{n}\mathrel{=}\mathrm{2}\mathbin{*}\Varid{n}\mathbin{+}\mathrm{4}\end{lcbr}\\\begin{lcbr}\Varid{g2}\;\mathrm{0}\mathrel{=}\mathrm{5}\\\Varid{g2}\;\Varid{n}\mathrel{=}\mathrm{2}\mathbin{*}\Varid{n}\mathbin{+}\mathrm{5}\end{lcbr}\end{lcbr}}
\ensuremath{\begin{lcbr}\Varid{g}\;\mathrm{0}\mathrel{=}\mathrm{6}\\\Varid{g}\;\Varid{n}\mathrel{=}\Varid{g1}\;(\Varid{n}\mathbin{-}\mathrm{1})\mathbin{*}\Varid{g2}\;(\Varid{n}\mathbin{-}\mathrm{1})\mathbin{*}\Varid{g}\;(\Varid{n}\mathbin{-}\mathrm{1})\end{lcbr}}
\end{eqnarray*}
Encontramos finalmente todas as operações e valores iniciais,
obtendo assim a nossa equação do \textit{picalc} neste modelo.

\begin{eqnarray*}
\start
\ensuremath{\begin{lcbr}\pi_{\mathit{calc}}\;\mathrm{0}\mathrel{=}\mathrm{2}\\\Varid{f}\;(\Varid{n}\mathbin{-}\mathrm{1})\mathbin{\uparrow}\mathrm{2}\mathbin{*}\Varid{t}\;(\Varid{n}\mathbin{-}\mathrm{1})\mathbin{/}\Varid{g}\;(\Varid{n}\mathbin{-}\mathrm{1})\mathbin{+}\pi_{\mathit{calc}}\;(\Varid{n}\mathbin{-}\mathrm{1})\end{lcbr}}
\end{eqnarray*}

Procede-se agora à construção de \textit{piloop} com as 
informações previamente calculadas (para simplificação, designaremos \(s = \textit{picalcRec}\)).

Seguem-se as operações relativas a \(s\):

\textbf{Operação de \(s\):}
\begin{hscode}\SaveRestoreHook
\column{B}{@{}>{\hspre}l<{\hspost}@{}}%
\column{17}{@{}>{\hspre}l<{\hspost}@{}}%
\column{E}{@{}>{\hspre}l<{\hspost}@{}}%
\>[B]{}\Varid{inicS}\mathrel{=}\mathrm{2}{}\<[E]%
\\
\>[B]{}\Varid{op\char95 S}\;\Varid{s}\;\Varid{f}\;\Varid{g}\;\Varid{t}\mathrel{=}{}\<[17]%
\>[17]{}\Varid{fromIntegral}\;(\Varid{f}\mathbin{\uparrow}\mathrm{2}\mathbin{*}\Varid{t})\mathbin{/}\Varid{fromIntegral}\;\Varid{g}\mathbin{+}\Varid{s}{}\<[E]%
\ColumnHook
\end{hscode}\resethooks

\textbf{Operação de \(g\):}
\begin{hscode}\SaveRestoreHook
\column{B}{@{}>{\hspre}l<{\hspost}@{}}%
\column{17}{@{}>{\hspre}l<{\hspost}@{}}%
\column{E}{@{}>{\hspre}l<{\hspost}@{}}%
\>[B]{}\Varid{inicG}\mathrel{=}\mathrm{6}{}\<[E]%
\\
\>[B]{}\Varid{op\char95 G}\;\Varid{g}\;\Varid{g1}\;\Varid{g2}\mathrel{=}{}\<[17]%
\>[17]{}\Varid{g}\mathbin{*}\Varid{g1}\mathbin{*}\Varid{g2}{}\<[E]%
\ColumnHook
\end{hscode}\resethooks

\textbf{Operação de \(t\):}
\begin{hscode}\SaveRestoreHook
\column{B}{@{}>{\hspre}l<{\hspost}@{}}%
\column{E}{@{}>{\hspre}l<{\hspost}@{}}%
\>[B]{}\Varid{inicT}\mathrel{=}\mathrm{4}{}\<[E]%
\\
\>[B]{}\Varid{op\char95 T}\;\Varid{t}\mathrel{=}\mathrm{2}\mathbin{*}\Varid{t}{}\<[E]%
\ColumnHook
\end{hscode}\resethooks

\textbf{Operação de \(f\):}
\begin{hscode}\SaveRestoreHook
\column{B}{@{}>{\hspre}l<{\hspost}@{}}%
\column{14}{@{}>{\hspre}l<{\hspost}@{}}%
\column{E}{@{}>{\hspre}l<{\hspost}@{}}%
\>[B]{}\Varid{inicF}\mathrel{=}\mathrm{1}{}\<[E]%
\\
\>[B]{}\Varid{op\char95 F}\;\Varid{f}\;\Varid{f2}\mathrel{=}{}\<[14]%
\>[14]{}\Varid{f}\mathbin{*}\Varid{f2}{}\<[E]%
\ColumnHook
\end{hscode}\resethooks

\textbf{Operação de \(f2\):}
\begin{hscode}\SaveRestoreHook
\column{B}{@{}>{\hspre}l<{\hspost}@{}}%
\column{E}{@{}>{\hspre}l<{\hspost}@{}}%
\>[B]{}\Varid{inicF2}\mathrel{=}\mathrm{2}{}\<[E]%
\\
\>[B]{}\Varid{op\char95 F2}\;\Varid{f}\;\Varid{f2}\mathrel{=}\succ \;\Varid{f2}{}\<[E]%
\ColumnHook
\end{hscode}\resethooks

\textbf{Operação de \(g1\):}
\begin{hscode}\SaveRestoreHook
\column{B}{@{}>{\hspre}l<{\hspost}@{}}%
\column{13}{@{}>{\hspre}l<{\hspost}@{}}%
\column{E}{@{}>{\hspre}l<{\hspost}@{}}%
\>[B]{}\Varid{inicG1}\mathrel{=}\mathrm{4}{}\<[E]%
\\
\>[B]{}\Varid{op\char95 G1}\;\Varid{g1}\mathrel{=}{}\<[13]%
\>[13]{}\Varid{g1}\mathbin{+}\mathrm{2}{}\<[E]%
\ColumnHook
\end{hscode}\resethooks

\textbf{Operação de \(g2\):}
\begin{hscode}\SaveRestoreHook
\column{B}{@{}>{\hspre}l<{\hspost}@{}}%
\column{13}{@{}>{\hspre}l<{\hspost}@{}}%
\column{E}{@{}>{\hspre}l<{\hspost}@{}}%
\>[B]{}\Varid{inicG2}\mathrel{=}\mathrm{5}{}\<[E]%
\\
\>[B]{}\Varid{op\char95 G2}\;\Varid{g2}\mathrel{=}{}\<[13]%
\>[13]{}\Varid{g2}\mathbin{+}\mathrm{2}{}\<[E]%
\ColumnHook
\end{hscode}\resethooks
Dispomos agora de todos os elementos necessários para construir o \textit{ciclo for}.

\begin{hscode}\SaveRestoreHook
\column{B}{@{}>{\hspre}l<{\hspost}@{}}%
\column{E}{@{}>{\hspre}l<{\hspost}@{}}%
\>[B]{}\Varid{worker}\mathrel{=}\for{\Varid{loop}}\ {\Varid{inic}}{}\<[E]%
\ColumnHook
\end{hscode}\resethooks
Os nossos valores iniciais vão ser os casos bases de cada função que definimos
\begin{hscode}\SaveRestoreHook
\column{B}{@{}>{\hspre}l<{\hspost}@{}}%
\column{E}{@{}>{\hspre}l<{\hspost}@{}}%
\>[B]{}\Varid{inic}\mathrel{=}(\Varid{inicS},\Varid{inicG},\Varid{inicT},\Varid{inicF},\Varid{inicF2},\Varid{inicG1},\Varid{inicG2}){}\<[E]%
\ColumnHook
\end{hscode}\resethooks

E as operações ja definimos também:
\begin{hscode}\SaveRestoreHook
\column{B}{@{}>{\hspre}l<{\hspost}@{}}%
\column{27}{@{}>{\hspre}l<{\hspost}@{}}%
\column{28}{@{}>{\hspre}l<{\hspost}@{}}%
\column{E}{@{}>{\hspre}l<{\hspost}@{}}%
\>[B]{}\Varid{loop}\;(\Varid{s},\Varid{g},\Varid{t},\Varid{f},\Varid{f2},\Varid{g1},\Varid{g2})\mathrel{=}{}\<[E]%
\\
\>[B]{}\hsindent{27}{}\<[27]%
\>[27]{}(\Varid{op\char95 S}\;\Varid{s}\;\Varid{f}\;\Varid{g}\;\Varid{t},{}\<[E]%
\\
\>[27]{}\hsindent{1}{}\<[28]%
\>[28]{}\Varid{op\char95 G}\;\Varid{g}\;\Varid{g1}\;\Varid{g2},{}\<[E]%
\\
\>[27]{}\hsindent{1}{}\<[28]%
\>[28]{}\Varid{op\char95 T}\;\Varid{t},{}\<[E]%
\\
\>[27]{}\hsindent{1}{}\<[28]%
\>[28]{}\Varid{op\char95 F}\;\Varid{f}\;\Varid{f2},{}\<[E]%
\\
\>[27]{}\hsindent{1}{}\<[28]%
\>[28]{}\Varid{op\char95 F2}\;\Varid{f}\;\Varid{f2},{}\<[E]%
\\
\>[27]{}\hsindent{1}{}\<[28]%
\>[28]{}\Varid{op\char95 G1}\;\Varid{g1},{}\<[E]%
\\
\>[27]{}\hsindent{1}{}\<[28]%
\>[28]{}\Varid{op\char95 G2}\;\Varid{g2}{}\<[E]%
\\
\>[27]{}\hsindent{1}{}\<[28]%
\>[28]{}){}\<[E]%
\ColumnHook
\end{hscode}\resethooks
Por fim, é necessário filtrar apenas o resultado, que corresponde à 
primeira componente do tuplo:
\begin{hscode}\SaveRestoreHook
\column{B}{@{}>{\hspre}l<{\hspost}@{}}%
\column{E}{@{}>{\hspre}l<{\hspost}@{}}%
\>[B]{}\Varid{wrapper}\;(\Varid{x},\anonymous ,\anonymous ,\anonymous ,\anonymous ,\anonymous ,\anonymous )\mathrel{=}\Varid{x}{}\<[E]%
\ColumnHook
\end{hscode}\resethooks

Finalmente, obtém-se a função construída com base num \textit{ciclo for}.

\begin{hscode}\SaveRestoreHook
\column{B}{@{}>{\hspre}l<{\hspost}@{}}%
\column{E}{@{}>{\hspre}l<{\hspost}@{}}%
\>[B]{}\pi_{\mathit{loop}}\mathrel{=}\Varid{wrapper}\comp \Varid{worker}{}\<[E]%
\ColumnHook
\end{hscode}\resethooks

\subsection*{Problema 4}

Começa-se por resolver o functor aplicado a \textit{Vec}, o qual terá os seguintes tipos:
\begin{eqnarray*}
\xymatrix@C=2cm{
&
    \ensuremath{\Conid{Vec}\;\Conid{A}}
    \ar[r]^-{\ensuremath{\mathsf{fmap}\;\Varid{f}}}
&
    \ensuremath{\Conid{Vec}\;\Conid{B}}
}
\end{eqnarray*}

Assim, chegamos facilmente à definição explícita deste functor, apresentada no seguinte diagrama:

\[
\centerline{
    \xymatrix@C=3cm{
        \ensuremath{\Conid{Vec}\;\Conid{A}}
        \ar[d]^-{\ensuremath{\Varid{outV}}}
    \\
        \ensuremath{(\Conid{A}\times\Conid{Int})}^*
        \ar[d]^-{\ensuremath{\map \;(\Varid{f}\times\Varid{id})}}
    \\
        \ensuremath{(\Conid{B}\times\Conid{Int})}^*
        \ar[d]^-{\ensuremath{\Conid{V}}}
    \\
        \ensuremath{\Conid{Vec}\;\Conid{B}}
    }
}
\]

Em \Haskell, a definição é dada por:

\begin{hscode}\SaveRestoreHook
\column{B}{@{}>{\hspre}l<{\hspost}@{}}%
\column{5}{@{}>{\hspre}l<{\hspost}@{}}%
\column{E}{@{}>{\hspre}l<{\hspost}@{}}%
\>[B]{}\mathbf{instance}\;\Conid{Functor}\;\Conid{Vec}\;\mathbf{where}{}\<[E]%
\\
\>[B]{}\hsindent{5}{}\<[5]%
\>[5]{}\mathsf{fmap}\;\Varid{f}\mathrel{=}\Conid{V}\comp \map \;(\Varid{f}\times\Varid{id})\comp \Varid{outV}{}\<[E]%
\ColumnHook
\end{hscode}\resethooks

Deve agora definir-se o operador \((>>=)\) e a função \textit{return} desta \textit{Monad}. 
Para tal, será necessário dispor de \(\mu\) para facilitar os cálculos e 
garantir a completude do problema.

O nosso \(\mu\) terá, naturalmente, os seguintes tipos:

\begin{eqnarray*}
\xymatrix@C=2cm{
&
    \ensuremath{\Conid{Vec}\;(\Conid{Vec}\;\Conid{A})}
    \ar[r]^-{\mu}
&
    \ensuremath{\Conid{Vec}\;\Conid{A}}
}
\end{eqnarray*}
Iniciaremos, então, a resolução do diagrama, nomeando a nossa função \(\mu\) como:
\textit{miuV}:
\[
\centerline{
\xymatrix@C=3cm@R=1.5cm{
        \ensuremath{\Conid{Vec}\;(\Conid{Vec}\;\Conid{A})}
            \ar[d]_-{\ensuremath{\Varid{outV}}}
    \\
        \ensuremath{(\Conid{Vec}\;\Conid{A}\times\Conid{Int})}^*
            \ar[d]_-{\ensuremath{(\Varid{outV}\times\Varid{id})}}
            \ar@/^2pc/[ddr]^-{\ensuremath{\Varid{concatMap}\;(\Varid{miuaux}\comp (\Varid{outV}\times\Varid{id}))}}
    \\
        \ensuremath{((\Conid{A}\times\Conid{Int})}^*\ensuremath{\times\Conid{Int})}^*
            \ar[d]_-{\ensuremath{\map \;\Varid{miuaux}}}
    \\
        \ensuremath{(\Conid{A}\times\Conid{Int})}^{*^*}
            \ar[r]_-{\ensuremath{\Varid{concat}}}
    &
        (\ensuremath{\Conid{A}\times\Conid{Int}})^*
            \ar[d]_-{\ensuremath{\Conid{V}}}
    \\
    &
        \ensuremath{\Conid{Vec}\;\Conid{A}}
    }
}
\]

Será igualmente necessário definir o \textit{miuaux}, que 
realizará a multiplicação de todos os elementos do vetor associado ao número.

\begin{hscode}\SaveRestoreHook
\column{B}{@{}>{\hspre}l<{\hspost}@{}}%
\column{E}{@{}>{\hspre}l<{\hspost}@{}}%
\>[B]{}\Varid{miuaux}\mathbin{::}([\mskip1.5mu (\Varid{a},\Conid{Int})\mskip1.5mu],\Conid{Int})\to [\mskip1.5mu (\Varid{a},\Conid{Int})\mskip1.5mu]{}\<[E]%
\ColumnHook
\end{hscode}\resethooks

\begin{eqnarray*}
    \xymatrix@C=3cm{
        \ensuremath{(\Conid{A}\times\Conid{Int})}^* \ensuremath{\times\Conid{Int}}
            \ar[d]_-{\ensuremath{\Varid{miuaux}}}
            \ar[r]^-{\ensuremath{\conj{\Varid{id}}{\length }\times\Varid{id}}}
    &
        \ensuremath{((\Conid{A}\times\Conid{Int})}^* \ensuremath{\times\N_0)\times\Conid{Int}}
            \ar[r]^-{\ensuremath{\Varid{assocr}}}
    &
        \ensuremath{(\Conid{A}\times\Conid{Int})}^* \ensuremath{\times(\N_0\times\Conid{Int})}
            \ar[d]^{\ensuremath{(\Varid{id}\times\uncurry{\Varid{replicate}})}}
    \\
        \ensuremath{(\Conid{A}\times\Conid{Int})}^*
    &
        \ensuremath{((\Conid{A}\times\Conid{Int})\times\Conid{Int})}^*
        \ar[l]^-{\ensuremath{\map \;((\Varid{id}\times\uncurry{(\mathbin{*})})\comp \Varid{assocr})}}
    &
         \ensuremath{(\Conid{A}\times\Conid{Int})}^* \ensuremath{\times\Conid{Int}}^*
            \ar[l]^-{\ensuremath{\uncurry{\Varid{zip}}}}
    }
\end{eqnarray*}

Procedeu-se à definição das duas funções em \Haskell.

\begin{hscode}\SaveRestoreHook
\column{B}{@{}>{\hspre}l<{\hspost}@{}}%
\column{E}{@{}>{\hspre}l<{\hspost}@{}}%
\>[B]{}\Varid{miuaux}\mathrel{=}\map \;((\Varid{id}\times\uncurry{(\mathbin{*})})\comp \Varid{assocr})\comp \uncurry{\Varid{zip}}\comp (\Varid{id}\times\uncurry{\Varid{replicate}})\comp \Varid{assocr}\comp (\conj{\Varid{id}}{\length }\times\Varid{id}){}\<[E]%
\\[\blanklineskip]%
\>[B]{}\Varid{miuV}\mathbin{::}\Conid{Vec}\;(\Conid{Vec}\;\Varid{a})\to \Conid{Vec}\;\Varid{a}{}\<[E]%
\\
\>[B]{}\Varid{miuV}\mathrel{=}\Conid{V}\comp \Varid{concatMap}\;(\Varid{miuaux}\comp (\Varid{outV}\times\Varid{id}))\comp \Varid{outV}{}\<[E]%
\ColumnHook
\end{hscode}\resethooks

Com o \textit{miuV} definido, é possível calcular facilmente \((x >>= f)\) de
 acordo com a fórmula (88) do formulário de \cp{Cálculo de Programas}.

\begin{eqnarray*}
\start
\ensuremath{\Varid{x}\bind \Varid{f}\mathrel{=}(}\mu\ensuremath{\comp \Conid{T}\;\Varid{f})\;\Varid{x}}
\just\equiv{\ensuremath{\Varid{substituindo}\;\Varid{pelo}\;\Varid{o}\;\Varid{nosso}\;\Varid{contexto}}}
\ensuremath{\Varid{x}\bind \Varid{f}\mathrel{=}(\Varid{miuV}\comp \mathsf{fmap}\;\Varid{f})\;\Varid{x}}
\end{eqnarray*}

Falta ainda calcular a função \textit{return}.
 Sabemos que \(\mu_V \circ \textit{return} = \textit{id}\);
  portanto, a definição de \textit{return} será:

\begin{eqnarray*}
\start
\ensuremath{\Varid{return}\mathrel{=}\Conid{V}\comp \Varid{singl}\comp \conj{\Varid{id}}{\underline{\mathrm{1}}}}
\end{eqnarray*}

Associa-se o número 1, dado que este é o elemento
neutro da multiplicação; assim, ao aplicar \(\mu_V\),
todos os elementos são multiplicados por 1, resultando
no mesmo vetor. Expressam-se em \Haskell\ o \textit{binding}
e o \textit{return} da seguinte forma:


\begin{hscode}\SaveRestoreHook
\column{B}{@{}>{\hspre}l<{\hspost}@{}}%
\column{4}{@{}>{\hspre}l<{\hspost}@{}}%
\column{E}{@{}>{\hspre}l<{\hspost}@{}}%
\>[B]{}\mathbf{instance}\;\Conid{Monad}\;\Conid{Vec}\;\mathbf{where}{}\<[E]%
\\
\>[B]{}\hsindent{4}{}\<[4]%
\>[4]{}\Varid{x}\bind \Varid{f}\mathrel{=}\Varid{miuV}\;(\mathsf{fmap}\;\Varid{f}\;\Varid{x}){}\<[E]%
\\
\>[B]{}\hsindent{4}{}\<[4]%
\>[4]{}\Varid{return}\mathrel{=}\Conid{V}\comp \Varid{singl}\comp \conj{\Varid{id}}{\underline{\mathrm{1}}}{}\<[E]%
\ColumnHook
\end{hscode}\resethooks

%----------------- Índice remissivo (exige makeindex) -------------------------%

\printindex

%----------------- Bibliografia (exige bibtex) --------------------------------%

\bibliographystyle{plain}
\bibliography{cp2425t}

%----------------- Fim do documento -------------------------------------------%
\end{document}
